\documentclass[11pt, english, fleqn, DIV=15, headinclude, BCOR=1cm]{scrartcl}

\usepackage[bibatend]{../header}

\usepackage{lastpage}
\usepackage{multicol}
\usepackage{simplewick}
\usepackage{slashed}
\usepackage{subcaption}
\usepackage{tikzsymbols}
\usepackage{stmaryrd}

\usepackage{../my-boxes}

\newcommand\timeorder{\mathscr T}
\newcommand\normorder{\mathscr N}
\newcommand\eye{\mat 1_4}
\newcommand\myslash[1]{\underline{\slashed{\vec{#1}}}}
\newcommand\ensemble[1]{\llbracket #1 \rrbracket}
\newcommand\Ensemble[1]{\left\llbracket #1 \right\rrbracket}
\newcommand\sump{\sideset{}{'}\sum}

\hypersetup{
    pdftitle=
}

\graphicspath{{build/}}

\newcounter{totalpoints}
\newcommand\punkte[1]{#1\addtocounter{totalpoints}{#1}}

\newcounter{problemset}
\setcounter{problemset}{7}

\subject{physics617 -- Theoretical Condensed Matter Physics}
\ihead{physics617 -- Problem Set \arabic{problemset}}

\title{Problem Set \arabic{problemset}}

\newcommand\thegroup{Tutor: Ramsés Sánchez}

\publishers{\thegroup}
\ofoot{\thegroup}

\author{
    Martin Ueding \\ \small{\href{mailto:mu@martin-ueding.de}{mu@martin-ueding.de}}
}
\ifoot{Martin Ueding}

\ohead{\rightmark}

\begin{document}

\maketitle

\vspace{3ex}

\begin{center}
    \begin{tabular}{rrr}
        \toprule
        Problem & Achieved points & Possible points \\
        \midrule
        \nameref{homework:1} & & \punkte{5} \\
        \nameref{homework:2} & & \punkte{10} \\
        \nameref{homework:3} & & \punkte{15} \\
        \midrule
        Total & & \arabic{totalpoints} \\
        \bottomrule
    \end{tabular}
\end{center}

\vspace{3ex}

\begin{center}
    \begin{small}
        This document consists of \pageref{LastPage} pages.
    \end{small}
\end{center}

\section{Green's function of an impurity immersed in a fermionic bath}
\label{homework:1}

\section{Yet another pair of frequency summations}
\label{homework:2}


\section{Wick's theorem for non-operators}
\label{homework:3}

The location of the impurities is defined as
\[
    \rho_{\vec q} := \sum_i \exp(- \iup \vec q \cdot \vec R_i) \,.
\]

\subsection{Ensemble averages}

The definition of the ensemble average given in the lecture is
\[
    \ensemble G := \int \sbr{\prod_{i=1}^N \dif^3 R_i} G(\vec R_1, \dotsc, \vec R_N) \,.
\]
The notation is a bit different here. First I use some other brackets. There
are double angle brackets in \texttt{MnSymbol} but that clashes with
\texttt{amssymb} and I do not want to get those to work right now. Those are
taken from \texttt{stmaryrd} which plays nice with \texttt{amssymb}. Then I
like to have function evaluation with round brackets no matter what the
arguments are. From a programming perspective I find it illogical that the
function arguments should be enclosed in square brackets if one of them is
another function. And because the theorist convinced me that writing the
differential forms (or measure or integration variable) up front makes more
sense, I will do that here as well.

It is important that the integration only applies to the actual values of $\vec
R$ used. Otherwise they would create infinite multiplicative factors which is
not the desired outcome. With that in mind we can start to evaluate some
averages.
\begin{align*}
    \ensemble{\rho_{\vec q}}
    &= \Ensemble{\sum_i \exp(- \iup \vec q \cdot \vec R_i)} \\
    \intertext{%
        The ensemble average is linear so we can pull out the sum.
    }
    &= \sum_i \Ensemble{\exp(- \iup \vec q \cdot \vec R_i)} \\
    \intertext{%
        Each of the terms is a functional of only one of the $\vec R$ now.
        Therefore we only need one integration.
    }
    &= \sum_i \int \dif^3 R_i \, \exp(- \iup \vec q \cdot \vec R_i)
    \intertext{%
        This integration just gives a $\delta$-distribution (perhaps with
        another factor of $2 \piup$?) and we have
    }
    &= \sum_i \delta(\vec q) \,.
    \intertext{%
        All the $N$ summands are equal, therefore this reduces to
    }
    &= N \delta(\vec q) \,.
\end{align*}
This matches the expression given in the lecture. There, both $n$ and $N$ are
used for the same thing I assume, so I will go with $N$ for everything.

The two-density expression can be computed similarly. We start with
\begin{align*}
    \ensemble{\rho_{\vec q_1} \rho_{\vec q_2}}
    &=: \ensemble{\rho_1 \rho_2}
    \intertext{%
        where we have introduced some shorthand notation. Next we write out the
        sums and have
    }
    &= \sum_{i, j} \Ensemble{\exp(- \iup \vec q_1 \cdot \vec R_i) \exp(- \iup
    \vec q_2 \cdot \vec R_j)} \,.
    \intertext{%
        The cases $i = j$ and $i \neq j$ have to be treated differently. We
        introduce the notation $\sum'$ which means that only unique elements
        are summed over. With that notation we have
    }
    &= \sum_{i} \Ensemble{\exp(- \iup [\vec q_1 + \vec q_2] \cdot \vec R_i)}
    +
    \sump_{i, j} \Ensemble{\exp(- \iup \vec q_1 \cdot \vec R_i) \exp(- \iup
    \vec q_2 \cdot \vec R_j)}
    \,.
    \intertext{%
        The first summand is something we have just calculated, it is just $N
        \delta(\vec q_1 + \vec q_2)$. In the second summand, we have two
        distinct integrals.
    }
    &= N \delta(\vec q_1 + \vec q_2)
    +
    \sump_{i, j}
    \int \dif^3 R_i \, \exp(- \iup \vec q_1 \cdot \vec R_i)
    \int \dif^3 R_j \, \exp(- \iup \vec q_2 \cdot \vec R_j)
    \intertext{%
        Each of the integrals gives a $\delta$-distribution.
    }
    &= N \delta(\vec q_1 + \vec q_2)
    + \sump_{i, j} \delta(\vec q_1) \delta(\vec q_2)
    \intertext{%
        There are $N[N-1] = N^2 - N$ summands in the second term. In the
        lecture, it was shown to be $N^2$, though. Perhaps it is because $N \gg
        1$ and therefore this can be approximated to be $N^2$ directly? We will
        just use this argument to obtain the result in the lecture on
        2015-12-01 which is the canonical result.
    }
    &= N \delta(\vec q_1 + \vec q_2)
    + N^2 \delta(\vec q_1) \delta(\vec q_2)
\end{align*}

From here the systematics are clear, there are just more combinations in the
case of three terms. We introduce even more shorthand notation. We write
\begin{align*}
    \ensemble{\rho_1 \rho_2 \rho_3}
    &= \sum_{i, j, k} \Ensemble{\rho_1^i \rho_2^j \rho_3^k}
    \intertext{%
        where the upper indices are the index of the $\vec R$ that appears in
        the exponential.
    }
\end{align*}

\subsection{Feynman diagrams}

Finally I can create some Feynman diagrams with Ti\emph{k}Z for this course as
well! \Laughey The \emph{Advanced Quantum Field Theory} course has lots of
opportunities to make those. Now you will not be spared with them either.
\Winkey

Table~\ref{tab:single} shows the Feynman diagrams for the scattering on a
single impurity. Multiple of those can be combined when multiple impurities are
considered in a scattering process.

\begin{table}
    \centering
    \begin{tabular}{cc}
        \toprule
        Term & Feynman Diagram \\
        \midrule
        $\delta(\vec q_1)$ & \includegraphics{one} \\
        \midrule
        $\delta(\vec q_1 + \vec q_2)$ & \includegraphics{two} \\
        \midrule
        $\delta(\vec q_1 + \vec q_2 + \vec q_3)$ & \includegraphics{three} \\
        \bottomrule
    \end{tabular}
    \caption{%
        Feynman diagrams with scattering on a single impurity. The impurity is
        denoted with a box as Ti\emph{k}Z does not seem to have a pre-made
        shape which resembles the cross used in the lecture.
    }
    \label{tab:single}
\end{table}

\begin{figure}
    \centering
    \includegraphics{iij}
    \caption{%
        Feynman diagram for scattering three times on two impurities. This
        diagrams corresponds to the quantity $\ensemble{\rho_1^i \rho_2^i
        \rho_3^k}$ as defined in the text.
    }
    \label{fig:iij}
\end{figure}

\subsection{$n$-th order}

\end{document}

% vim: spell spelllang=en tw=79
