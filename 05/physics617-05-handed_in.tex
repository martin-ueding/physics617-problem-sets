\documentclass[11pt, english, fleqn, DIV=15, headinclude, BCOR=1cm]{scrartcl}

\usepackage[bibatend]{../header}

\usepackage{lastpage}
\usepackage{multicol}
\usepackage{simplewick}
\usepackage{slashed}
\usepackage{subcaption}

\usepackage[tikz]{mdframed}
\newmdtheoremenv[%
    backgroundcolor=black!5,
    innertopmargin=\topskip,
    splittopskip=\topskip,
    skipabove=\topskip,
    roundcorner=4pt,
]{theorem}{Theorem}[section]

\usepackage{../my-boxes}

\newcommand\timeorder{\mathscr T}
\newcommand\normorder{\mathscr N}
\newcommand\eye{\mat 1_4}
\newcommand\myslash[1]{\underline{\slashed{\vec{#1}}}}

\hypersetup{
    pdftitle=
}

\graphicspath{{build/}}

\newcounter{totalpoints}
\newcommand\punkte[1]{#1\addtocounter{totalpoints}{#1}}

\newcounter{problemset}
\setcounter{problemset}{4}

\subject{physics617 -- Theoretical Condensed Matter Physics}
\ihead{physics617 -- Problem Set \arabic{problemset}}

\title{Problem Set \arabic{problemset}}

\newcommand\thegroup{Tutor: Ramsés Sánchez}

\publishers{\thegroup}
\ofoot{\thegroup}

\author{
    Martin Ueding \\ \small{\href{mailto:mu@martin-ueding.de}{mu@martin-ueding.de}}
}
\ifoot{Martin Ueding}

\ohead{\rightmark}

\begin{document}

\maketitle

\vspace{3ex}

\begin{center}
    \begin{tabular}{rrr}
        \toprule
        Problem & Achieved points & Possible points \\
        \midrule
        \nameref{homework:1} & & \punkte{15} \\
        \nameref{homework:2} & & \punkte{15} \\
        \midrule
        Total & & \arabic{totalpoints} \\
        \bottomrule
    \end{tabular}
\end{center}

\vspace{3ex}

\begin{center}
    \begin{small}
        This document consists of \pageref{LastPage} pages.
    \end{small}
\end{center}

\section{Matsubara Green's function}
\label{homework:1}

\subsection{Translation in imaginary time}

Today I feel a bit mathematical, therefore I will structure this in theorems
and proofs.

\begin{theorem}[Time translation]
    \label{the:time-translation}
    Let $G$ be the Matsubara Green's function. Then we have a time translation
    invariance, i.e.\ $G(\tau, \tau') = G(\tau - \tau', 0)$.
\end{theorem}

\begin{proof}
    In general we have
    \begin{align*}
        G(\tau, \tau')
        &= - \bracket{T_\tau A(\tau) B(\tau')} \,.
        \intertext{%
            We assume $\tau > \tau'$. Then we have the special case of
        }
        &= - \bracket{A(\tau) B(\tau')} \,,
        \intertext{%
            where only the first part of the time ordering is needed. We have
            to expand the thermodynamic average and the time evolution
            operators. We obtain
        }
        &= - \frac1Z \tr\del{\eup^{\beta H} \eup^{\tau H} A(\tau) \eup^{-\tau H}
        \eup^{\tau' H} B \eup^{-\tau' H}} \,.
        \intertext{%
            Now we us the cyclicity of the trace and move the time evolution
            operators around. Specifically we move the last one to the front.
            Then we have
        }
        &= - \frac1Z \tr\del{\eup^{-\tau' H} \eup^{\beta H} \eup^{\tau H} A(\tau) \eup^{-\tau H}
        \eup^{\tau' H} B} \,.
        \intertext{%
            Next we can commute the first two exponentials as they are
            functions of the Hamiltonian only. The expression then assumes the
            form
        }
        &= - \frac1Z \tr\del{\eup^{\beta H} \eup^{-\tau' H} \eup^{\tau H} A(\tau) \eup^{-\tau H}
        \eup^{\tau' H} B} \,.
        \intertext{%
            Here we can join the exponentials suggestively to obtain
        }
        &= - \frac1Z \tr\del{\eup^{\beta H} \eup^{[\tau - \tau'] H} A(\tau)
        \eup^{-[\tau - \tau'] H}
        B} \,,
        \intertext{%
            which is
        }
        &= G(\tau - \tau') \,.
    \end{align*}
    Therefore the Green's function as defined is imaginary time translational
    invariant. It only depends on the difference of the times, not the actual
    times themselves.
\end{proof}

\subsection{Periodicity}

\begin{theorem}
    If $-\beta < \tau < 0$ then $G(\tau) = - \epsilon G(\tau + \beta)$.
\end{theorem}

\begin{proof}
    We will start with the assumption and show that it actually holds.
    \begin{align*}
        G(\tau) &= - \epsilon G(\tau + \beta)
        \intertext{%
            We insert the explicit definitions.
        }
        - \bracket{A(\tau)B} \Theta(\tau) + \epsilon \bracket{B A(\tau)}
        \Theta(-\tau) &= \epsilon \bracket{A(\tau + \beta) B} \Theta(\tau +
        \beta) - \epsilon^2 \bracket{BA(\tau + \beta)} \Theta(-\tau-\beta)
        \intertext{%
            We have $-\beta < \tau < 0$. Therefore only one term on each side
            actually is selected by the step function.
            We can even drop the step function then.
        }
        \epsilon \bracket{B A(\tau)}
        &= \epsilon \bracket{A(\tau + \beta) B}
        \intertext{%
            Then we write out the thermodynamic average.
        }
        \epsilon \tr\del{\eup^{-\beta H} B \eup^{\tau H} A \eup^{-\tau H}}
        &= \epsilon \tr\del{\eup^{-\beta H} \eup^{\beta H} \eup^{\tau H} A
        \eup^{-\tau H} \eup^{-\beta H} B}
        \intertext{%
            The first two exponentials on the right side cancel.
        }
        \epsilon \tr\del{\eup^{-\beta H} B \eup^{\tau H} A \eup^{-\tau H}}
        &= \epsilon \tr\del{\eup^{\tau H} A
        \eup^{-\tau H} \eup^{-\beta H} B}
        \intertext{%
            And then we use the cyclic property of the trace to move the last
            two terms up front.
        }
        \epsilon \tr\del{\eup^{-\beta H} B \eup^{\tau H} A \eup^{-\tau H}}
        &= \epsilon \tr\del{\eup^{-\beta H} B \eup^{\tau H} A \eup^{-\tau H}}
    \end{align*}
    Then the relation holds.
\end{proof}

\begin{question}
    In the very last step I have exchanged the order of $A$ and $B$. The
    cyclicity of the trace can be shown easily when it is done with (finite)
    matrices and one just uses the two indices per matrix and contracts them
    round-robin. Since everything is contracted, one can shuffle the terms. Do
    the individual elements of $A$ and $B$ simply commute then? In general they
    have an (anti)commutator which should be zero at different times (which we
    have here). However, if the elements anticommute, then the overall
    calculation should have another minus sign. From the desired expression it
    seems that the elements of $A$ and $B$ do commute. The cyclicity of the
    trace seems to be completely independent of the nature of the operators in
    the trace. Is that the case, can I always use the cyclic property even with
    badly non-(anti)commuting operators?
\end{question}

\subsection{Requirements for Equation~(2)}

We can look at Equation~(3) where the Green's function is decomposed into
Fourier modes. It reads
\[
    G(\tau) = \frac 1 \beta \sum_l \tilde G(\omega_l) \exp(- \iup \omega_l
    \tau) \,.
\]
If we shift $\tau$ by $\beta$

\section{Green's function of an impurity immersed in a fermionic bath}
\label{homework:2}

\end{document}

% vim: spell spelllang=en tw=79
