\documentclass[11pt, english, fleqn, DIV=15, headinclude, BCOR=1cm]{scrartcl}

\usepackage[bibatend]{../header}

\usepackage{lastpage}
\usepackage{multicol}
\usepackage{simplewick}
\usepackage{slashed}
\usepackage{subcaption}

\usepackage[tikz]{mdframed}
\newmdtheoremenv[%
    backgroundcolor=black!5,
    innertopmargin=\topskip,
    splittopskip=\topskip,
    skipabove=\topskip,
    roundcorner=4pt,
]{theorem}{Theorem}[section]

\usepackage{../my-boxes}

\newcommand\timeorder{\mathscr T}
\newcommand\normorder{\mathscr N}
\newcommand\eye{\mat 1_4}
\newcommand\myslash[1]{\underline{\slashed{\vec{#1}}}}

\hypersetup{
    pdftitle=
}

\graphicspath{{build/}}

\newcounter{totalpoints}
\newcommand\punkte[1]{#1\addtocounter{totalpoints}{#1}}

\newcounter{problemset}
\setcounter{problemset}{5}

\subject{physics617 -- Theoretical Condensed Matter Physics}
\ihead{physics617 -- Problem Set \arabic{problemset}}

\title{Problem Set \arabic{problemset}}

\newcommand\thegroup{Tutor: Ramsés Sánchez}

\publishers{\thegroup}
\ofoot{\thegroup}

\author{
    Martin Ueding \\ \small{\href{mailto:mu@martin-ueding.de}{mu@martin-ueding.de}}
}
\ifoot{Martin Ueding}

\ohead{\rightmark}

\begin{document}

\maketitle

\vspace{3ex}

\begin{center}
    \begin{tabular}{rrr}
        \toprule
        Problem & Achieved points & Possible points \\
        \midrule
        \nameref{homework:1} & & \punkte{15} \\
        \nameref{homework:2} & & \punkte{15} \\
        \midrule
        Total & & \arabic{totalpoints} \\
        \bottomrule
    \end{tabular}
\end{center}

\vspace{3ex}

\begin{center}
    \begin{small}
        This document consists of \pageref{LastPage} pages.
    \end{small}
\end{center}

\section{Matsubara Green's function}
\label{homework:1}

\subsection{Translation in imaginary time}

Today I feel a bit mathematical, therefore I will structure this in theorems
and proofs.

\begin{theorem}[Time translation]
    \label{the:time-translation}
    Let $G$ be the Matsubara Green's function. Then we have a time translation
    invariance, i.e.\ $G(\tau, \tau') = G(\tau - \tau', 0)$.
\end{theorem}

\begin{proof}
    In general we have
    \begin{align*}
        G(\tau, \tau')
        &= - \bracket{T_\tau A(\tau) B(\tau')} \,.
        \intertext{%
            We assume $\tau > \tau'$. Then we have the special case of
        }
        &= - \bracket{A(\tau) B(\tau')} \,,
        \intertext{%
            where only the first part of the time ordering is needed. We have
            to expand the thermodynamic average and the time evolution
            operators. We obtain
        }
        &= - \frac1Z \tr\del{\eup^{\beta H} \eup^{\tau H} A(\tau) \eup^{-\tau H}
        \eup^{\tau' H} B \eup^{-\tau' H}} \,.
        \intertext{%
            Now we us the cyclicity of the trace and move the time evolution
            operators around. Specifically we move the last one to the front.
            Then we have
        }
        &= - \frac1Z \tr\del{\eup^{-\tau' H} \eup^{\beta H} \eup^{\tau H} A(\tau) \eup^{-\tau H}
        \eup^{\tau' H} B} \,.
        \intertext{%
            Next we can commute the first two exponentials as they are
            functions of the Hamiltonian only. The expression then assumes the
            form
        }
        &= - \frac1Z \tr\del{\eup^{\beta H} \eup^{-\tau' H} \eup^{\tau H} A(\tau) \eup^{-\tau H}
        \eup^{\tau' H} B} \,.
        \intertext{%
            Here we can join the exponentials suggestively to obtain
        }
        &= - \frac1Z \tr\del{\eup^{\beta H} \eup^{[\tau - \tau'] H} A(\tau)
        \eup^{-[\tau - \tau'] H}
        B} \,,
        \intertext{%
            which is
        }
        &= G(\tau - \tau') \,.
    \end{align*}
    Therefore the Green's function as defined is imaginary time translational
    invariant. It only depends on the difference of the times, not the actual
    times themselves.
\end{proof}

\subsection{Periodicity}

\begin{theorem}
    If $-\beta < \tau < 0$ then $G(\tau) = - \epsilon G(\tau + \beta)$.
\end{theorem}

\begin{proof}
    We will start with the assumption and show that it actually holds.
    \begin{align*}
        G(\tau) &= - \epsilon G(\tau + \beta)
        \intertext{%
            We insert the explicit definitions.
        }
        - \bracket{A(\tau)B} \Theta(\tau) + \epsilon \bracket{B A(\tau)}
        \Theta(-\tau) &= \epsilon \bracket{A(\tau + \beta) B} \Theta(\tau +
        \beta) - \epsilon^2 \bracket{BA(\tau + \beta)} \Theta(-\tau-\beta)
        \intertext{%
            We have $-\beta < \tau < 0$. Therefore only one term on each side
            actually is selected by the step function.
            We can even drop the step function then.
        }
        \epsilon \bracket{B A(\tau)}
        &= \epsilon \bracket{A(\tau + \beta) B}
        \intertext{%
            Then we write out the thermodynamic average.
        }
        \epsilon \tr\del{\eup^{-\beta H} B \eup^{\tau H} A \eup^{-\tau H}}
        &= \epsilon \tr\del{\eup^{-\beta H} \eup^{\beta H} \eup^{\tau H} A
        \eup^{-\tau H} \eup^{-\beta H} B}
        \intertext{%
            The first two exponentials on the right side cancel.
        }
        \epsilon \tr\del{\eup^{-\beta H} B \eup^{\tau H} A \eup^{-\tau H}}
        &= \epsilon \tr\del{\eup^{\tau H} A
        \eup^{-\tau H} \eup^{-\beta H} B}
        \intertext{%
            And then we use the cyclic property of the trace to move the last
            two terms up front.
        }
        \epsilon \tr\del{\eup^{-\beta H} B \eup^{\tau H} A \eup^{-\tau H}}
        &= \epsilon \tr\del{\eup^{-\beta H} B \eup^{\tau H} A \eup^{-\tau H}}
    \end{align*}
    Then the relation holds.
\end{proof}

\begin{question}
    In the very last step I have exchanged the order of $A$ and $B$. The
    cyclicity of the trace can be shown easily when it is done with (finite)
    matrices and one just uses the two indices per matrix and contracts them
    round-robin. Since everything is contracted, one can shuffle the terms. Do
    the individual elements of $A$ and $B$ simply commute then? In general they
    have an (anti)commutator which should be zero at different times (which we
    have here). However, if the elements anticommute, then the overall
    calculation should have another minus sign. From the desired expression it
    seems that the elements of $A$ and $B$ do commute. The cyclicity of the
    trace seems to be completely independent of the nature of the operators in
    the trace. Is that the case, can I always use the cyclic property even with
    badly non-(anti)commuting operators?
\end{question}

\needspace{4cm}
\subsection{Requirements for Equation~(2)}

\begin{theorem}
    The frequencies are given by
    \[
        \omega_l = \frac 1 \beta
        \begin{cases}
            2 l \piup & \epsilon = 1 \,, \\
            [2 l + 1] \piup & \epsilon = -1 \,.
        \end{cases}
    \]
\end{theorem}

\begin{proof}
    We can look at Equation~(3) where the Green's function is decomposed into
    Fourier modes. It reads
    \[
        G(\tau) = \frac 1 \beta \sum_l \tilde G(\omega_l) \exp(- \iup \omega_l
        \tau) \,.
    \]
    Assume $\epsilon = -1$. Then we expect from the previous problem that we do
    not occur a sign change with a shift. If we shift $\tau$ by $\beta$, then
    the exponential obtains an additional factor $\exp(-\iup \omega_l \beta)$.
    The exponent must be an integer multiple of $2 \piup \iup$ such that we do
    not incur a phase factor. Therefore we have
    \[
        \omega_l = \frac{2 l \piup}{\beta}
    \]
    for the case $\epsilon = -1$. Now look at $\epsilon = 1$. Then we do need a
    phase factor of $-1$ which means that $\omega_l$ must be larger by
    $\piup/\beta$. Both cases together are what the conjecture says.
\end{proof}

\subsection{Fourier transform}

We are asked to rewrite the Green's function
\begin{align*}
    \tilde G(\omega_l)
    &= \int_0^\beta \dif \tau \exp(\iup \omega_l \tau) G(\tau) \,.
    \intertext{%
        We insert the explicit form of $G$. We then have
    }
    &= \int_0^\beta \dif \tau \exp(\iup \omega_l \tau)
    \sbr{- \bracket{A(\tau) B} \Theta(\tau) + \epsilon \bracket{BA(\tau)}
    \Theta(-\tau)}
    \,.
    \intertext{%
        In the domain of integration $\tau$ is always in the interval $(0,
        \beta)$ as we can omit zero measure intervals in the Riemann integral.
        This means that only the first summand with $\Theta(\tau)$ will
        contribute to this integral. This means that the remainder is
    }
    &= - \int_0^\beta \dif \tau \exp(\iup \omega_l \tau)
    \bracket{A(\tau) B}
    \,.
    \intertext{%
        Then we insert the thermodynamic average as well as the time evolution
        explicitly and obtain
    }
    &= - \frac{1}{Z} \int_0^\beta \dif \tau \exp(\iup \omega_l \tau)
    \tr\del{\eup^{-\beta H} \eup^{\tau H} A \eup^{-\tau H} B}
    \,.
    \intertext{%
        However, this is not very helpful as is. We need to insert some
        eigenstates of $n$ and $m$ in oder to make this useful. Our
        intermediate state then is
    }
    &= - \frac{1}{Z} \sum_{m,n} \int_0^\beta \dif \tau \exp(\iup \omega_l \tau)
    \tr\del{\eup^{-\beta H} \eup^{\tau H} \ket n \bra n A \ket m \bra m \eup^{-\tau H} B}
    \,,
    \intertext{%
        where we will evaluate the energy expectation values of the inserted
        states. We chose to let the first two exponentials act to the right. The
        last exponential acts to the left. This get us
    }
    &= - \frac{1}{Z} \sum_{m,n} \int_0^\beta \dif \tau \exp(\iup \omega_l \tau)
    \tr\del{\eup^{-\beta E_n} \eup^{\tau E_n} \ket n \bra n A \ket m \bra m
    \eup^{-\tau E_m} B}
    \,.
    \intertext{%
        Using the cyclicity of the trace we want to sandwich the operators into
        matrix elements. However, we will have to write the trace explicitly to
        see how exactly that is possible. The trace is just a sandwich of all
        states. We introduce another summation index, $k$. This gives us
    }
    &= - \frac{1}{Z} \sum_{k,m,n} \int_0^\beta \dif \tau \exp(\iup \omega_l \tau)
    \bra k \sbr{\eup^{-\beta E_n} \eup^{\tau E_n} \ket n \bra n A \ket m \bra m
    \eup^{-\tau E_m} B} \ket k
    \,.
    \intertext{%
        All the constant factors can be extracted and all the bra-kets combined
        to give nice compact terms:
    }
    &= - \frac{1}{Z} \sum_{k,m,n} \int_0^\beta \dif \tau \exp(\iup \omega_l \tau)
    \eup^{-\beta E_n} \eup^{\tau E_n} \eup^{-\tau E_m}
    \braket{k|n} \braket{n|A|m} \braket{m|B|k}
    \,.
    \intertext{%
        The first and last bra-ket is just a Kronecker symbol. We can eliminate
        the summation index $k$ using those directly. This simplifies the
        expression to
    }
    &= - \frac{1}{Z} \sum_{m,n} \int_0^\beta \dif \tau \exp(\iup \omega_l \tau)
    \eup^{-\beta E_n} \eup^{\tau E_n} \eup^{-\tau E_m}
    \braket{n|A|m} \braket{m|B|n}
    \,,
    \intertext{%
        where we can use the shorthands for the matrix elements to yield
    }
    &= - \frac{1}{Z} \sum_{m,n} A_{nm} B_{mn}
    \eup^{-\beta E_n}
    \int_0^\beta \dif \tau \exp(\iup \omega_l \tau)
    \eup^{\tau E_n} \eup^{-\tau E_m}
    \,.
    \intertext{%
        We have pulled the matrix elements up front directly as they do not
        depend on the integral any more. The one term that does not depend on
        $\tau$ is now also outside of the integral. We combine all the
        exponentials into one to make the integration easier. This then is
    }
    &= - \frac{1}{Z} \sum_{m,n} A_{nm} B_{mn}
    \eup^{-\beta E_n}
    \int_0^\beta \dif \tau \exp(\tau[E_n - E_m + \iup \omega_l])
    \,.
    \intertext{%
        This integral is actually simple and we can directly compute it. Our
        preliminary result is
    }
    &= - \frac{1}{Z} \sum_{m,n} A_{nm} B_{mn}
    \eup^{-\beta E_n}
    \frac{\exp(\beta[E_n - E_m + \iup \omega_l]) - 1}{E_n - E_m + \iup \omega_l}
    \,,
    \intertext{%
        which we will simplify to
    }
    &= - \frac{1}{Z} \sum_{m,n} A_{nm} B_{mn}
    \frac{\exp(\beta[- E_m + \iup \omega_l]) - \exp(-\beta E_n)}{E_n - E_m + \iup \omega_l}
    \,.
\end{align*}
That did not work out quite right. There is no $\epsilon$ term. We would have
gotten that one in case that the $\tau$ would be negative at some point. Also
we have never interchanged $A$ and $B$ which would also give a factor
$-\epsilon$. It seems that the integral consists of two parts which need to be
added up. Then the $\exp(\iup \beta \omega_l)$ would cancel perhaps and we
would obtain some factor of $- \epsilon$ along the way.

Maybe this can be done by using the time translation and making the integral
symmetric around zero and let it run in the interval $(- \beta, \beta)$. This
would make it symmetric in the positive and negative imaginary time and then we
would have the needed factors.

\subsection{Integration along real line}

\emph{Missing}

\section{Green's function of an impurity immersed in a fermionic bath}
\label{homework:2}

\subsection{Equations of motion}

\paragraph{Time derivative of $G$}

For the equation of motion we need the time derivative of both Green's
functions.
\begin{align*}
    \pd{}\tau G(\tau)
    &= \pd{}\tau \sbr{- \bracket{c(\tau) c^\dagger } \Theta(\tau) + \epsilon
    \bracket{c^\dagger c(\tau)}
    \Theta(-\tau)}
    \intertext{%
        The time derivative will act on both the thermodynamic average as well
        as the step function, giving four terms in total.
    }
    &= - \sbr{\pd{}\tau \bracket{c(\tau) c^\dagger }} \Theta(\tau)
    - \bracket{c(\tau) c^\dagger } \delta(\tau)
    + \epsilon \sbr{\pd{}\tau \bracket{c^\dagger c(\tau)}} \Theta(-\tau)
    - \epsilon \bracket{c^\dagger c(\tau)} \delta(-\tau)
    \intertext{%
        The thermodynamic average does not depend on the time, so we can just
        move the time derivative into the thermodynamic bracket.
    }
    &= - \Bracket{\pd{}\tau c(\tau) c^\dagger } \Theta(\tau)
    - \bracket{c(\tau) c^\dagger } \delta(\tau)
    + \epsilon \Bracket{\pd{}\tau c^\dagger c(\tau)} \Theta(-\tau)
    - \epsilon \bracket{c^\dagger c(\tau)} \delta(-\tau)
    \intertext{%
        As can be seen quickly by expressing $c(\tau)$ as $c$ and time
        evolution operators, the time derivative is given by the commutator
        with the Hamiltonian.
    }
    &= - \bracket{[H, c(\tau)] c^\dagger } \Theta(\tau)
    - \bracket{c(\tau) c^\dagger } \delta(\tau)
    + \epsilon \bracket{c^\dagger [H, c(\tau)]} \Theta(-\tau)
    - \epsilon \bracket{c^\dagger c(\tau)} \delta(-\tau)
\end{align*}

\paragraph{Time derivative of $F$}

An analogous time derivative also holds for the second Green's function:
\[
    \pd{}\tau F_{\vec k}(\tau)
    =
    - \bracket{[H, f_{\vec k}(\tau)] c^\dagger } \Theta(\tau)
    - \bracket{f_{\vec k}(\tau) c^\dagger } \delta(\tau)
    + \epsilon \bracket{c^\dagger [H, f_{\vec k}(\tau)]} \Theta(-\tau)
    - \epsilon \bracket{c^\dagger f_{\vec k}(\tau)} \delta(-\tau)
\]

\paragraph{Complete set of eigenstates}

Before it makes any sense to plug those time derivatives and the functions
themselves in the equations of motion, we need a way to set the states that are
annihilated by $c$ and $f_{\vec k}$ in relation with each other. The term
\[
    \sum_{\vec k} \sbr{c^\dagger f_{\vec k} + f_{\vec k}^\dagger c}
\]
in the Hamiltonian suggests that the overlap $\braket{c | \vec k}$ is non-zero,
the state $\ket c$ is a superposition of the momentum states. A complete set of
eigenstates should therefore be obtained by $\sum_{\vec k} \ket{\vec k}
\bra{\vec k} = \mat 1$.

\paragraph{Evaluation of $F$}

One of the thermodynamic averages that we encounter in the first equation of
motion is
\begin{align*}
    F_{\vec k}(\tau)
    &= \braket{f_{\vec k}(\tau) c^\dagger} \,. \\
    \intertext{%
        We expand the thermodynamic average and yield
    }
    &= \frac 1Z \sum_{\vec p} \braket{\vec p | \eup^{-\beta H} \eup^{\tau H}
    f_{\vec p} \eup^{-\tau H} c^\dagger | \vec p} \,.
    \intertext{%
        We let the Hamiltonians act to the left.
    }
    &= \frac 1Z \sum_{\vec p} \eup^{-\beta E_{\vec p}} \eup^{\tau E_{\vec p}}
    \braket{\vec p | f_{\vec k} \eup^{-\tau H} c^\dagger | \vec p}
    \intertext{%
        Now we can let $f_{\vec k}$ act to the left and see that it would
        create another state with momentum $\vec k$. If $\vec k = \vec p$, this
        would give zero as we are working with Fermions. If $\vec k \neq \vec
        p$ that would create a multi particle state which has a non-trivial
        energy. Therefore we insert another set of eigenstates here.
    }
    &= \frac 1Z \sum_{\vec p, \vec q} \eup^{-\beta E_{\vec p}} \eup^{\tau E_{\vec p}}
    \braket{\vec p | f_{\vec k} | \vec q}
    \braket{\vec q | \eup^{-\tau H} c^\dagger | \vec p}
    \intertext{%
        Now $f_{\vec k}$ can annihilate the state $\ket{\vec q}$ which will
        give us $\delta(\vec k - \vec q) \ket 0$. Then the overlap
        $\braket{\vec p | 0}$ will give
        0. Therefore we have
    }
    &= 0 \,.
\end{align*}
This cannot be right. Otherwise the Green's function would be zero all the
time.

The assumption that $\sum_{\vec k} \ket{\vec k} \bra{\vec k} = \mat 1$ is
wrong and there are states missing there. Perhaps the state $c^\dagger \ket 0$
must be included there as well. Since that state is supposed to have
non-vanishing overlap with the momentum states, it does seem redundant to
insert this state into the sum. However, since have a variable particle number,
we need the Fock space. Therefore we have to consider the states with arbitrary
particle number here in order to get the correct matrix elements. So going back
and inserting a set of two-particle states we have
\begin{align*}
    F_{\vec k}(\tau)
    &= \frac 1Z \sum_{\vec p, \vec q, \vec Q} \eup^{-\beta E_{\vec p}} \eup^{\tau E_{\vec p}}
    \braket{\vec p | f_{\vec k} | \vec q, \vec Q}
    \braket{\vec q, \vec Q | \eup^{-\tau H} c^\dagger | \vec p} \,.
    \intertext{%
        The case $\vec q = \vec Q$ will give the state $\ket{\vec q, \vec q}$
        which is just 0, so we do not need to worry about that explicitly. The
        annihilation $f_{\vec k}$ can make some sense now. It can annihilate
        either of the two states, so we have
    }
    &= \frac 1Z \sum_{\vec p, \vec q, \vec Q} \eup^{-\beta E_{\vec p}} \eup^{\tau E_{\vec p}}
    \bra{\vec p}
    \sbr{
        \delta(\vec k - \vec q) \ket{\vec Q}
        + \delta(\vec k - \vec Q) \ket{\vec q}
    }
    \braket{\vec q, \vec Q | \eup^{-\tau H} c^\dagger | \vec p} \,.
    \intertext{%
        We expand the whole term.
    }
    &= \frac 1Z \sum_{\vec p, \vec q, \vec Q} \eup^{-\beta E_{\vec p}} \eup^{\tau E_{\vec p}}
    \sbr{
        \braket{\vec p | \delta(\vec k - \vec q) | \vec Q}
        \braket{\vec q, \vec Q | \eup^{-\tau H} c^\dagger | \vec p}
        + \braket{\vec p | \delta(\vec k - \vec Q) | \vec q}
        \braket{\vec q, \vec Q | \eup^{-\tau H} c^\dagger | \vec p}
    }
    \intertext{%
        Then we can apply the $\delta$-distribution.
    }
    &= \frac 1Z \sum_{\vec p} \eup^{-\beta E_{\vec p}} \eup^{\tau E_{\vec p}}
    \sbr{
        \sum_{\vec Q}
        \braket{\vec p | \vec Q}
        \braket{\vec k, \vec Q | \eup^{-\tau H} c^\dagger | \vec p}
        +
        \sum_{\vec q}
        \braket{\vec p | \vec q}
        \braket{\vec q, \vec k | \eup^{-\tau H} c^\dagger | \vec p}
    }
    \intertext{%
        Here we can rename the dummy state $\vec q$ and $\vec Q$ in the
        respective terms.
    }
    &= \frac 1Z \sum_{\vec p, \vec q} \eup^{-\beta E_{\vec p}} \eup^{\tau E_{\vec p}}
    \braket{\vec p | \vec q}
    \sbr{
        \braket{\vec k, \vec q | \eup^{-\tau H} c^\dagger | \vec p}
        +
        \braket{\vec q, \vec k | \eup^{-\tau H} c^\dagger | \vec p}
    }
    \intertext{%
        The two terms are almost the same. They differ in the order, though.
        Therefore we have to interchange one of them and incur a minus sign.
        The overlap $\braket{\vec p | \vec q}$ is just $\delta(\vec p - \vec
        q)$. We remove the sum over $\vec q$ with that.
    }
    &= \frac 1Z \sum_{\vec p} \eup^{-\beta E_{\vec p}} \eup^{\tau E_{\vec p}}
    \sbr{
        \braket{\vec k, \vec p | \eup^{-\tau H} c^\dagger | \vec p}
        -
        \braket{\vec k, \vec p | \eup^{-\tau H} c^\dagger | \vec p}
    }
    \intertext{%
        \emph{If} the change of sign was correct, then this vanishes again.
    }
    &= 0
\end{align*}
I just cannot accept that $F_{\vec k}(\tau)$ vanishes for every $\vec k$ and
every $\tau$. It does not make any sense to have an always vanishing object.

It is pointless to continue without a way of setting $f_{\vec k}$ and $c$ in a
relation. The left and right side of the equation contain different operators
such that one cannot directly compare them. Although there are three pages of
stuff, none of that really brings me closer to show that the equations of
motion hold. Therefore I stop here and wait for next week's tutorial.

\subsection{Solutions}

We are asked to Fourier transform the equations of motion and then solve them.
But we do have the solutions in imaginary time already given, right? So in
principle one should be able to Fourier transform those to get the same result,
right?

The Fourier transform of the first equation is
\[
    \int \dif \tau \, \exp(- \iup \omega \tau) \sbr{\pd{}\tau + \epsilon_c}
    G(\tau) = \int \dif \tau \, \exp(- \iup \omega \tau)  \sbr{-\delta(\tau) -
    V \sum_{\vec k} F_{\vec k}(\tau)} \,.
\]
With the conventions made, a derivative can be moved using a partial
integration and just gives a factor $\iup \omega$. We therefore have
\[
    \epsilon_c + \iup \omega \tilde G(\omega) = -1 - V \sum_{\vec k} \tilde
    F_{\vec k}(\omega) \,.
\]

The same can be done for the other equation, that will yield
\[
    [\iup \omega + \epsilon(\vec k)] \tilde F_{\vec k}(\omega) = - V \tilde
    G(\omega) \,.
\]

Then we insert the equations for $\tilde F$ into the other equation and have
\begin{align*}
    [\epsilon_c + \iup \omega] \tilde G(\omega) &= - 1 - V \sum_{\vec k}
    \frac{- V \tilde G(\omega)}{\iup \omega + \epsilon(k)} \,.
    \intertext{%
        From here we bring all the $\tilde G$ dependent terms onto the other
        side.
    }
    \sbr{\epsilon_c + \iup \omega - \sum_{\vec k}
    \frac{V^2}{\iup \omega + \epsilon(k)}} \tilde G(\omega) &= - 1 \,.
    \intertext{%
        And as a last step we divide by that square bracket.
    }
    \tilde G(\omega) &= \sbr{- \epsilon_c - \iup \omega + \sum_{\vec k}
    \frac{V^2}{\iup \omega + \epsilon(k)}}^{-1}
\end{align*}
This is somewhat like in Equation~(10) on the problem set. I guess that another
sign convention for the Fourier transformation is used such that all the
$\iup\omega$ terms have the other sign. Changing those signs, it turns to the
right result:
\[
    \tilde G(\omega) = \sbr{- \epsilon_c + \iup \omega - \sum_{\vec k}
    \frac{V^2}{\iup \omega - \epsilon(k)}}^{-1} \,.
\]
So this was just a deviation in the Fourier transform, nothing more.

\end{document}

% vim: spell spelllang=en tw=79
