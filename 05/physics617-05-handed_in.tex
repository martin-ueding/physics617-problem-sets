\documentclass[11pt, english, fleqn, DIV=15, headinclude, BCOR=1cm]{scrartcl}

\usepackage[bibatend]{../header}

\usepackage{lastpage}
\usepackage{multicol}
\usepackage{simplewick}
\usepackage{slashed}
\usepackage{subcaption}

\usepackage[tikz]{mdframed}
\newmdtheoremenv[%
    backgroundcolor=black!5,
    innertopmargin=\topskip,
    splittopskip=\topskip,
    skipabove=\topskip,
    roundcorner=4pt,
]{theorem}{Theorem}[section]

\usepackage{../my-boxes}

\newcommand\timeorder{\mathscr T}
\newcommand\normorder{\mathscr N}
\newcommand\eye{\mat 1_4}
\newcommand\myslash[1]{\underline{\slashed{\vec{#1}}}}

\hypersetup{
    pdftitle=
}

\graphicspath{{build/}}

\newcounter{totalpoints}
\newcommand\punkte[1]{#1\addtocounter{totalpoints}{#1}}

\newcounter{problemset}
\setcounter{problemset}{4}

\subject{physics617 -- Theoretical Condensed Matter Physics}
\ihead{physics617 -- Problem Set \arabic{problemset}}

\title{Problem Set \arabic{problemset}}

\newcommand\thegroup{Tutor: Ramsés Sánchez}

\publishers{\thegroup}
\ofoot{\thegroup}

\author{
    Martin Ueding \\ \small{\href{mailto:mu@martin-ueding.de}{mu@martin-ueding.de}}
}
\ifoot{Martin Ueding}

\ohead{\rightmark}

\begin{document}

\maketitle

\vspace{3ex}

\begin{center}
    \begin{tabular}{rrr}
        \toprule
        Problem & Achieved points & Possible points \\
        \midrule
        \nameref{homework:1} & & \punkte{15} \\
        \nameref{homework:2} & & \punkte{15} \\
        \midrule
        Total & & \arabic{totalpoints} \\
        \bottomrule
    \end{tabular}
\end{center}

\vspace{3ex}

\begin{center}
    \begin{small}
        This document consists of \pageref{LastPage} pages.
    \end{small}
\end{center}

\section{Matsubara Green's function}
\label{homework:1}

\subsection{Translation in imaginary time}

Today I feel a bit mathematical, therefore I will structure this in theorems
and proofs.

\begin{theorem}[Time translation]
    \label{the:time-translation}
    Let $G$ be the Matsubara Green's function. Then we have a time translation
    invariance, i.e.\ $G(\tau, \tau') = G(\tau - \tau', 0)$.
\end{theorem}

\begin{proof}
    In general we have
    \begin{align*}
        G(\tau, \tau')
        &= - \bracket{T_\tau A(\tau) B(\tau')} \,.
        \intertext{%
            We assume $\tau > \tau'$. Then we have the special case of
        }
        &= - \bracket{A(\tau) B(\tau')} \,,
        \intertext{%
            where only the first part of the time ordering is needed. We have
            to expand the thermodynamic average and the time evolution
            operators. We obtain
        }
        &= - \frac1Z \tr\del{\eup^{\beta H} \eup^{\tau H} A(\tau) \eup^{-\tau H}
        \eup^{\tau' H} B \eup^{-\tau' H}} \,.
        \intertext{%
            Now we us the cyclicity of the trace and move the time evolution
            operators around. Specifically we move the last one to the front.
            Then we have
        }
        &= - \frac1Z \tr\del{\eup^{-\tau' H} \eup^{\beta H} \eup^{\tau H} A(\tau) \eup^{-\tau H}
        \eup^{\tau' H} B} \,.
        \intertext{%
            Next we can commute the first two exponentials as they are
            functions of the Hamiltonian only. The expression then assumes the
            form
        }
        &= - \frac1Z \tr\del{\eup^{\beta H} \eup^{-\tau' H} \eup^{\tau H} A(\tau) \eup^{-\tau H}
        \eup^{\tau' H} B} \,.
        \intertext{%
            Here we can join the exponentials suggestively to obtain
        }
        &= - \frac1Z \tr\del{\eup^{\beta H} \eup^{[\tau - \tau'] H} A(\tau)
        \eup^{-[\tau - \tau'] H}
        B} \,,
        \intertext{%
            which is
        }
        &= G(\tau - \tau') \,.
    \end{align*}
    Therefore the Green's function as defined is imaginary time translational
    invariant. It only depends on the difference of the times, not the actual
    times themselves.
\end{proof}

\subsection{Periodicity}

\begin{theorem}
    If $-\beta < \tau < 0$ then $G(\tau) = - \epsilon G(\tau + \beta)$.
\end{theorem}

\begin{proof}
    We will start with the assumption and show that it actually holds.
    \begin{align*}
        G(\tau) &= - \epsilon G(\tau + \beta)
        \intertext{%
            We insert the explicit definitions.
        }
        - \bracket{A(\tau)B} \Theta(\tau) + \epsilon \bracket{B A(\tau)}
        \Theta(-\tau) &= \epsilon \bracket{A(\tau + \beta) B} \Theta(\tau +
        \beta) - \epsilon^2 \bracket{BA(\tau + \beta)} \Theta(-\tau-\beta)
        \intertext{%
            We have $-\beta < \tau < 0$. Therefore only one term on each side
            actually is selected by the step function.
            We can even drop the step function then.
        }
        \epsilon \bracket{B A(\tau)}
        &= \epsilon \bracket{A(\tau + \beta) B}
        \intertext{%
            Then we write out the thermodynamic average.
        }
        \epsilon \tr\del{\eup^{-\beta H} B \eup^{\tau H} A \eup^{-\tau H}}
        &= \epsilon \tr\del{\eup^{-\beta H} \eup^{\beta H} \eup^{\tau H} A
        \eup^{-\tau H} \eup^{-\beta H} B}
        \intertext{%
            The first two exponentials on the right side cancel.
        }
        \epsilon \tr\del{\eup^{-\beta H} B \eup^{\tau H} A \eup^{-\tau H}}
        &= \epsilon \tr\del{\eup^{\tau H} A
        \eup^{-\tau H} \eup^{-\beta H} B}
        \intertext{%
            And then we use the cyclic property of the trace to move the last
            two terms up front.
        }
        \epsilon \tr\del{\eup^{-\beta H} B \eup^{\tau H} A \eup^{-\tau H}}
        &= \epsilon \tr\del{\eup^{-\beta H} B \eup^{\tau H} A \eup^{-\tau H}}
    \end{align*}
    Then the relation holds.
\end{proof}

\begin{question}
    In the very last step I have exchanged the order of $A$ and $B$. The
    cyclicity of the trace can be shown easily when it is done with (finite)
    matrices and one just uses the two indices per matrix and contracts them
    round-robin. Since everything is contracted, one can shuffle the terms. Do
    the individual elements of $A$ and $B$ simply commute then? In general they
    have an (anti)commutator which should be zero at different times (which we
    have here). However, if the elements anticommute, then the overall
    calculation should have another minus sign. From the desired expression it
    seems that the elements of $A$ and $B$ do commute. The cyclicity of the
    trace seems to be completely independent of the nature of the operators in
    the trace. Is that the case, can I always use the cyclic property even with
    badly non-(anti)commuting operators?
\end{question}

\needspace{4cm}
\subsection{Requirements for Equation~(2)}

\begin{theorem}
    The frequencies are given by
    \[
        \omega_l = \frac 1 \beta
        \begin{cases}
            2 l \piup & \epsilon = 1 \,, \\
            [2 l + 1] \piup & \epsilon = -1 \,.
        \end{cases}
    \]
\end{theorem}

\begin{proof}
    We can look at Equation~(3) where the Green's function is decomposed into
    Fourier modes. It reads
    \[
        G(\tau) = \frac 1 \beta \sum_l \tilde G(\omega_l) \exp(- \iup \omega_l
        \tau) \,.
    \]
    Assume $\epsilon = -1$. Then we expect from the previous problem that we do
    not occur a sign change with a shift. If we shift $\tau$ by $\beta$, then
    the exponential obtains an additional factor $\exp(-\iup \omega_l \beta)$.
    The exponent must be an integer multiple of $2 \piup \iup$ such that we do
    not incur a phase factor. Therefore we have
    \[
        \omega_l = \frac{2 l \piup}{\beta}
    \]
    for the case $\epsilon = -1$. Now look at $\epsilon = 1$. Then we do need a
    phase factor of $-1$ which means that $\omega_l$ must be larger by
    $\piup/\beta$. Both cases together are what the conjecture says.
\end{proof}

\subsection{Fourier transform}

We are asked to rewrite the Green's function
\begin{align*}
    \tilde G(\omega_l)
    &= \int_0^\beta \dif \tau \exp(\iup \omega_l \tau) G(\tau) \,.
    \intertext{%
        We insert the explicit form of $G$. We then have
    }
    &= \int_0^\beta \dif \tau \exp(\iup \omega_l \tau)
    \sbr{- \bracket{A(\tau) B} \Theta(\tau) + \epsilon \bracket{BA(\tau)}
    \Theta(-\tau)}
    \,.
    \intertext{%
        In the domain of integration $\tau$ is always in the interval $(0,
        \beta)$ as we can omit zero measure intervals in the Riemann integral.
        This means that only the first summand with $\Theta(\tau)$ will
        contribute to this integral. This means that the remainder is
    }
    &= - \int_0^\beta \dif \tau \exp(\iup \omega_l \tau)
    \bracket{A(\tau) B}
    \,.
    \intertext{%
        Then we insert the thermodynamic average as well as the time evolution
        explicitly and obtain
    }
    &= - \frac{1}{Z} \int_0^\beta \dif \tau \exp(\iup \omega_l \tau)
    \tr\del{\eup^{-\beta H} \eup^{\tau H} A \eup^{-\tau H} B}
    \,.
    \intertext{%
        However, this is not very helpful as is. We need to insert some
        eigenstates of $n$ and $m$ in oder to make this useful. Our
        intermediate state then is
    }
    &= - \frac{1}{Z} \sum_{m,n} \int_0^\beta \dif \tau \exp(\iup \omega_l \tau)
    \tr\del{\eup^{-\beta H} \eup^{\tau H} \ket n \bra n A \ket m \bra m \eup^{-\tau H} B}
    \,,
    \intertext{%
        where we will evaluate the energy expectation values of the inserted
        states. We chose to let the first two exponentials act to the right. The
        last exponential acts to the left. This get us
    }
    &= - \frac{1}{Z} \sum_{m,n} \int_0^\beta \dif \tau \exp(\iup \omega_l \tau)
    \tr\del{\eup^{-\beta E_n} \eup^{\tau E_n} \ket n \bra n A \ket m \bra m
    \eup^{-\tau E_m} B}
    \,.
    \intertext{%
        Using the cyclicity of the trace we want to sandwich the operators into
        matrix elements. However, we will have to write the trace explicitly to
        see how exactly that is possible. The trace is just a sandwich of all
        states. We introduce another summation index, $k$. This gives us
    }
    &= - \frac{1}{Z} \sum_{k,m,n} \int_0^\beta \dif \tau \exp(\iup \omega_l \tau)
    \bra k \sbr{\eup^{-\beta E_n} \eup^{\tau E_n} \ket n \bra n A \ket m \bra m
    \eup^{-\tau E_m} B} \ket k
    \,.
    \intertext{%
        All the constant factors can be extracted and all the bra-kets combined
        to give nice compact terms:
    }
    &= - \frac{1}{Z} \sum_{k,m,n} \int_0^\beta \dif \tau \exp(\iup \omega_l \tau)
    \eup^{-\beta E_n} \eup^{\tau E_n} \eup^{-\tau E_m}
    \braket{k|n} \braket{n|A|m} \braket{m|B|k}
    \,.
    \intertext{%
        The first and last bra-ket is just a Kronecker symbol. We can eliminate
        the summation index $k$ using those directly. This simplifies the
        expression to
    }
    &= - \frac{1}{Z} \sum_{m,n} \int_0^\beta \dif \tau \exp(\iup \omega_l \tau)
    \eup^{-\beta E_n} \eup^{\tau E_n} \eup^{-\tau E_m}
    \braket{n|A|m} \braket{m|B|n}
    \,,
    \intertext{%
        where we can use the shorthands for the matrix elements to yield
    }
    &= - \frac{1}{Z} \sum_{m,n} A_{nm} B_{mn}
    \eup^{-\beta E_n}
    \int_0^\beta \dif \tau \exp(\iup \omega_l \tau)
    \eup^{\tau E_n} \eup^{-\tau E_m}
    \,.
    \intertext{%
        We have pulled the matrix elements up front directly as they do not
        depend on the integral any more. The one term that does not depend on
        $\tau$ is now also outside of the integral. We combine all the
        exponentials into one to make the integration easier. This then is
    }
    &= - \frac{1}{Z} \sum_{m,n} A_{nm} B_{mn}
    \eup^{-\beta E_n}
    \int_0^\beta \dif \tau \exp(\tau[E_n - E_m + \iup \omega_l])
    \,.
    \intertext{%
        This integral is actually simple and we can directly compute it. Our
        preliminary result is
    }
    &= - \frac{1}{Z} \sum_{m,n} A_{nm} B_{mn}
    \eup^{-\beta E_n}
    \frac{\exp(\beta[E_n - E_m + \iup \omega_l]) - 1}{E_n - E_m + \iup \omega_l}
    \,,
    \intertext{%
        which we will simplify to
    }
    &= - \frac{1}{Z} \sum_{m,n} A_{nm} B_{mn}
    \frac{\exp(\beta[- E_m + \iup \omega_l]) - \exp(-\beta E_n)}{E_n - E_m + \iup \omega_l}
    \,.
\end{align*}
That did not work out quite right. There is no $\epsilon$ term. We would have
gotten that one in case that the $\tau$ would be negative at some point. Also
we have never interchanged $A$ and $B$ which would also give a factor
$-\epsilon$. It seems that the integral consists of two parts which need to be
added up. Then the $\exp(\iup \beta \omega_l)$ would cancel perhaps and we
would obtain some factor of $- \epsilon$ along the way.

Maybe this can be done by using the time translation and making the integral
symmetric around zero and let it run in the interval $(- \beta, \beta)$. This
would make it symmetric in the positive and negative imaginary time and then we
would have the needed factors.

\subsection{Integration along real line}

\section{Green's function of an impurity immersed in a fermionic bath}
\label{homework:2}

\subsection{Equations of motion}

For the equation of motion we need the time derivative of both Green's
functions.
\begin{align*}
    \pd{}\tau G(\tau)
    &= \pd{}\tau \sbr{- \bracket{c(\tau) c^\dagger } \Theta(\tau) + \epsilon
    \bracket{c^\dagger c(\tau)}
    \Theta(-\tau)}
    \intertext{%
        The time derivative will act on both the thermodynamic average as well
        as the step function, giving four terms in total.
    }
    &= - \sbr{\pd{}\tau \bracket{c(\tau) c^\dagger }} \Theta(\tau)
    - \bracket{c(\tau) c^\dagger } \delta(\tau)
    + \epsilon \sbr{\pd{}\tau \bracket{c^\dagger c(\tau)}} \Theta(-\tau)
    - \epsilon \bracket{c^\dagger c(\tau)} \delta(-\tau)
    \intertext{%
        The thermodynamic average does not depend on the time, so we can just
        move the time derivative into the thermodynamic bracket.
    }
    &= - \Bracket{\pd{}\tau c(\tau) c^\dagger } \Theta(\tau)
    - \bracket{c(\tau) c^\dagger } \delta(\tau)
    + \epsilon \Bracket{\pd{}\tau c^\dagger c(\tau)} \Theta(-\tau)
    - \epsilon \bracket{c^\dagger c(\tau)} \delta(-\tau)
    \intertext{%
        As can be seen quickly by expressing $c(\tau)$ as $c$ and time
        evolution operators, the time derivative is given by the commutator
        with the Hamiltonian.
    }
    &= - \bracket{[H, c(\tau)] c^\dagger } \Theta(\tau)
    - \bracket{c(\tau) c^\dagger } \delta(\tau)
    + \epsilon \bracket{c^\dagger [H, c(\tau)]} \Theta(-\tau)
    - \epsilon \bracket{c^\dagger c(\tau)} \delta(-\tau)
\end{align*}

\subsection{Solutions}

\end{document}

% vim: spell spelllang=en tw=79
