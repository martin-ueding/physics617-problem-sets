\documentclass[11pt, english, fleqn, DIV=15, headinclude, BCOR=1cm]{scrartcl}

\usepackage[bibatend]{../header}

\usepackage{lastpage}
\usepackage{multicol}
\usepackage{simplewick}
\usepackage{slashed}
\usepackage{subcaption}
\usepackage{tikz-feynman}
%\usepackage{tikzsymbols}
\usepackage{stmaryrd}

\usepackage{../my-boxes}

\newcommand\timeorder{\mathscr T}
\newcommand\normorder{\mathscr N}
\newcommand\eye{\mat 1_4}
\newcommand\myslash[1]{\underline{\slashed{\vec{#1}}}}
\newcommand\ensemble[1]{\llbracket #1 \rrbracket}
\newcommand\Ensemble[1]{\left\llbracket #1 \right\rrbracket}
\newcommand\sump{\sideset{}{'}\sum}

\hypersetup{
    pdftitle=
}

\graphicspath{{build/}}

\newcounter{totalpoints}
\newcommand\punkte[1]{#1\addtocounter{totalpoints}{#1}}

\newcounter{problemset}
\setcounter{problemset}{8}

\subject{physics617 -- Theoretical Condensed Matter Physics}
\ihead{physics617 -- Problem Set \arabic{problemset}}

\title{Problem Set \arabic{problemset}}

\newcommand\thegroup{Tutor: Ramsés Sánchez}

\publishers{\thegroup}
\ofoot{\thegroup}

\author{
    Martin Ueding \\ \small{\href{mailto:mu@martin-ueding.de}{mu@martin-ueding.de}}
}
\ifoot{Martin Ueding}

\ohead{\rightmark}

\begin{document}

\maketitle

\vspace{3ex}

\begin{center}
    \begin{tabular}{rrr}
        \toprule
        Problem & Achieved points & Possible points \\
        \midrule
        \nameref{homework:1} & & \punkte{15} \\
        \nameref{homework:2} & & \punkte{10} \\
        \midrule
        Total & & \arabic{totalpoints} \\
        \bottomrule
    \end{tabular}
\end{center}

\vspace{3ex}

\begin{center}
    \begin{small}
        This document consists of \pageref{LastPage} pages.
    \end{small}
\end{center}

\section{First order perturbation theory}
\label{homework:1}

\subsection{Wick's theorem}

We are given a two-point correlation function with first order in $V$. This
means that there are two interaction vertices at $\vec x_1$ and $\vec x_2$.
There is a shorthand often used by \textcite{Bruus/Many-Body} that is $(1) :=
(\vec x_1, \tau_1)$. For $(\vec x, \tau)$ we will use $(a)$ and for $(\vec x',
\tau')$ we use $(b)$. Then we can write the given expression as
\[
    \frac 12 \int \dif 1 \dif 2 \, V(\vec x_1, \vec x_2), \delta(\tau_1, -
    \tau_2) \Bracket{
        T_\tau
        \psi_a \psi_1 \psi_2 \psi^\dagger_2 \psi^\dagger_1 \psi^\dagger_b
    } \,.
\]

Wick's theorem then tells us that we can write this as
\[
    \frac12 \int \dif 1 \dif 2 \, V(\vec x_1, \vec x_2), \delta(\tau_1, -
    \tau_2)
    \begin{vmatrix}
        G^{(0)}(b, a) & G^{(0)}(b, 1) & G^{(0)}(b, 2) \\
        G^{(0)}(1, a) & G^{(0)}(1, 1) & G^{(0)}(1, 2) \\
        G^{(0)}(2, a) & G^{(0)}(2, 1) & G^{(0)}(2, 2) \\
    \end{vmatrix} \,,
\]
where we now have a determinant as we deal with fermions. The interaction is
always between the points $(1)$ and $(2)$. We have external vertices $(a)$ and
$(b)$. The determinant will give us six terms of the style
\[
    \frac12 \int \dif 1 \dif 2 \, V(\vec x_1, \vec x_2), \delta(\tau_1, -
    \tau_2)
    G^{(0)}(b, a)
    G^{(0)}(1, 1)
    G^{(0)}(2, 2)
\]
from Sarrus's rule.

\subsection{Diagrams}

Writing only the arguments of the three Green's function we have the six
diagrams. The first one is a disconnected one,
\begin{align*}
    (b, a) (1, 1) (2, 2)
    &=
    \feynmandiagram[spring electrical layout, horizontal=a to b, inline=(a.base)]
    {
        a[particle=$a$] --[fermion] b[particle=$b$];
    };
    \times
    \feynmandiagram[spring electrical layout, scale=0.5, horizontal=1 to 2,
    inline=(1.south)]
    {
        1 --[photon] 2;
        1 --[fermion, half left] 1x;
        1 --[fermion, half right] 1x;
        2 --[fermion, half left] 2x;
        2 --[fermion, half right] 2x;
    };
    \,.
    \\
    %%%%%%%%%%%%%%%%%%%%%%%%%%%%%%%%%%%%%%%%%%%%%%%%%%%%%%%%%%%%%%%%%%%%%%
    \intertext{%
        Then there are two equivalent diagrams with a Fock term,
    }
    (b, 1) (1, 2) (2, a)
    &=
    \feynmandiagram[spring electrical layout, horizontal=a to b, inline=(a.base)]
    {
        a[particle=$a$] --[fermion] 2[label=270:2] --[fermion] 1[label=270:1] --[fermion] b[particle=$b$];
        2 --[photon, half left] 1;
    }; \,, \\
    %%%%%%%%%%%%%%%%%%%%%%%%%%%%%%%%%%%%%%%%%%%%%%%%%%%%%%%%%%%%%%%%%%%%%%
    (b, 2) (2, 1) (1, a)
    &=
    \feynmandiagram[spring electrical layout, horizontal=a to b, inline=(a.base)]
    {
        a[particle=$a$] --[fermion] 2[label=270:1] --[fermion] 1[label=270:2] --[fermion] b[particle=$b$];
        2 --[photon, half left] 1;
    }; \,. \\
    %%%%%%%%%%%%%%%%%%%%%%%%%%%%%%%%%%%%%%%%%%%%%%%%%%%%%%%%%%%%%%%%%%%%%%
    \intertext{%
        The fourth is again disconnected.
    }
    (b, a) (1, 2) (2, 1)
    &=
    \feynmandiagram[horizontal=a to b, inline=(a.base)]
    {
        a[particle=$a$] --[fermion] b[particle=$b$];
    };
    \times
    \feynmandiagram[horizontal=1 to 2, inline=(1.base)]
    {
        1[particle=$1$] --[photon] 2[particle=$2$];
        1 --[fermion, half left] 2;
        2 --[fermion, half left] 1;
    }; \\
    %%%%%%%%%%%%%%%%%%%%%%%%%%%%%%%%%%%%%%%%%%%%%%%%%%%%%%%%%%%%%%%%%%%%%%
    \intertext{%
        There are two diagrams with a tadpole subdiagram (Hartree term):
    }
    (b, 1) (1, a), (2, 2)
    &=
    \feynmandiagram[spring electrical layout, horizontal=a to b, scale=0.5,
    inline=(a.base)]
    {
        a[particle=$a$] --[fermion] 1 --[fermion] b[particle=$b$];
        1 --[photon] 2;
        2 --[fermion, half left] 2x;
        2x --[fermion, half left] 2;
    }; \\
    %%%%%%%%%%%%%%%%%%%%%%%%%%%%%%%%%%%%%%%%%%%%%%%%%%%%%%%%%%%%%%%%%%%%%%
    (b, 2) (2, a), (1, 1)
    &=
    \feynmandiagram[spring electrical layout, horizontal=a to b, scale=0.5,
    inline=(a.base)]
    {
        a[particle=$a$] --[fermion] 1 --[fermion] b[particle=$b$];
        1 --[photon] 2;
        2 --[fermion, half left] 2x;
        2x --[fermion, half left] 2;
    };
    %%%%%%%%%%%%%%%%%%%%%%%%%%%%%%%%%%%%%%%%%%%%%%%%%%%%%%%%%%%%%%%%%%%%%%
\end{align*}

Those are all the diagrams with exact first order in $V$.

\section{First order perturbation theory, part 2}
\label{homework:2}




\end{document}

% vim: spell spelllang=en tw=79
