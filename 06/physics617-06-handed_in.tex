\documentclass[11pt, english, fleqn, DIV=15, headinclude, BCOR=1cm]{scrartcl}

\usepackage[bibatend]{../header}

\usepackage{lastpage}
\usepackage{multicol}
\usepackage{simplewick}
\usepackage{slashed}
\usepackage{subcaption}
\usepackage{tikzsymbols}

\usepackage{../my-boxes}

\newcommand\timeorder{\mathscr T}
\newcommand\normorder{\mathscr N}
\newcommand\eye{\mat 1_4}
\newcommand\myslash[1]{\underline{\slashed{\vec{#1}}}}

\hypersetup{
    pdftitle=
}

\graphicspath{{build/}}

\newcounter{totalpoints}
\newcommand\punkte[1]{#1\addtocounter{totalpoints}{#1}}

\newcounter{problemset}
\setcounter{problemset}{6}

\subject{physics617 -- Theoretical Condensed Matter Physics}
\ihead{physics617 -- Problem Set \arabic{problemset}}

\title{Problem Set \arabic{problemset}}

\newcommand\thegroup{Tutor: Ramsés Sánchez}

\publishers{\thegroup}
\ofoot{\thegroup}

\author{
    Martin Ueding \\ \small{\href{mailto:mu@martin-ueding.de}{mu@martin-ueding.de}}
}
\ifoot{Martin Ueding}

\ohead{\rightmark}

\begin{document}

\maketitle

\vspace{3ex}

\begin{center}
    \begin{tabular}{rrr}
        \toprule
        Problem & Achieved points & Possible points \\
        \midrule
        \nameref{homework:1} & & \punkte{15} \\
        \nameref{homework:2} & & \punkte{10} \\
        \nameref{homework:3} & & \punkte{0} \\
        \midrule
        Total & & \arabic{totalpoints} \\
        \bottomrule
    \end{tabular}
\end{center}

\vspace{3ex}

\begin{center}
    \begin{small}
        This document consists of \pageref{LastPage} pages.
    \end{small}
\end{center}

\section{Free-phonon Green's function}
\label{homework:1}

\subsection{Ladder operators}

\newcommand\omegaq{\omega_{\vec q}}
\newcommand\omegamq{\omega_{-\vec q}}

We have
\[
    a_{\vec q}^\dagger = \eup^{\tau H} a_{\vec q}^\dagger \eup^{-\tau H} \,.
\]
To simplify this, we want to use the Baker-Campbell-Hausdorff formula with $A =
\tau H$ and $B = a_{\vec q}^\dagger$. The commutator of $A$ and $B$ is
\begin{align*}
    [A, B]
    &= [\tau H, a_{\vec q}^\dagger] \,.
    \intertext{%
        We insert the explicit form of the Hamiltonian.
    }
    &= \tau \sum_{\vec k} \omega_{\vec k}
    [a_{\vec k}^\dagger a_{\vec k}, a_{\vec q}^\dagger]
    \intertext{%
        Since the raising operator always commutes with itself, we can pull
        that out of the commutator already.
    }
    &= \tau \sum_{\vec k} \omega_{\vec k}
    a_{\vec k}^\dagger [a_{\vec k}, a_{\vec q}^\dagger]
    \intertext{%
        This commutator is just $\delta(\vec k - \vec q)$.
    }
    &= \tau \omegaq a_{\vec q}^\dagger
\end{align*}

We see that the commutator $[A, B]$ is just $\tau \omegaq B$ again. Therefore
the next commutator is $[A, [A, B]] = \tau\omegaq [A, B]$. We can factor out
the remaining $B$ and yield
$\exp(\tau \omegaq) B$ from all the terms. Our result therefore is
\[
    a_{\vec q}^\dagger(\tau) = \exp(\tau \omegaq) a_{\vec q}^\dagger \,.
\]

The derivation for $a_{\vec q}$ is very similar. The sign in the exponent
changes as the two terms in the commutator are now in the wrong order, they
need to be switched first before the canonical commutation relation can be
applied. Therefore we have
\[
    a_{\vec q}(\tau) = \exp(- \tau \omegaq) a_{\vec q} \,.
\]

\subsection{Green's function}

First we have to build up $A(\vec q, \tau)$ from the time dependent ladder
operators that we have just derived. Given is
\begin{align*}
    A(\vec q, \tau)
    &= \eup^{\tau H} [a_{\vec q} + a_{- \vec q}^\dagger] \eup^{- \tau H} \,.
    \intertext{%
        There we insert the expressions for the ladder operators.
    }
    &= \exp(- \tau \omegaq) a_{\vec q} + \exp(\tau \omegamq) a_{-\vec q}^\dagger
\end{align*}

Having that, we can compute the Green's function. The definition is
\begin{align*}
    D(\vec q, \tau)
    &= - \bracket{T_\tau A(\vec q, t) A(- \vec q, 0)} \,.
    \intertext{%
        Then we insert $A$.
    }
    &= - \bracket{T_\tau
    \sbr{\exp(- \tau \omegaq) a_{\vec q} + \exp(\tau \omegamq) a_{-\vec q}^\dagger}
    [a_{-\vec q} + a_{\vec q}^\dagger]}
    \intertext{%
        We expand the time ordering.
    }
    &= - \bracket{\sbr{\exp(- \tau \omegaq) a_{\vec q} + \exp(\tau \omegamq) a_{-\vec q}^\dagger}
    [a_{-\vec q} + a_{\vec q}^\dagger]} \Theta(\tau)
    \\&\quad
    - \bracket{[a_{-\vec q} + a_{\vec q}^\dagger]
    \sbr{\exp(- \tau \omegaq) a_{\vec q} + \exp(\tau \omegamq) a_{-\vec
    q}^\dagger}} \Theta(-\tau)
    \intertext{%
        Only number operators contribute to the traces. Therefore only two
        products for each time branch do anything.
    }
    &= - \bracket{\exp(- \tau \omegaq) a_{\vec q} a_{\vec q}^\dagger
    + \exp(\tau \omegamq) a_{-\vec q}^\dagger a_{-\vec q}} \Theta(\tau)
    \\&\quad
    - \bracket{\exp(\tau \omegamq) a_{-\vec q} a_{-\vec q}^\dagger
    + \exp(- \tau \omegaq) a_{\vec q}^\dagger a_{\vec q}} \Theta(-\tau)
    \intertext{%
        The exponentials can be taken out of the traces.
    }
    &= - \exp(- \tau \omegaq) \bracket{a_{\vec q} a_{\vec q}^\dagger} \Theta(\tau)
    - \exp(\tau \omegamq) \bracket{a_{-\vec q}^\dagger a_{-\vec q}} \Theta(\tau)
    \\&\quad
    - \exp(\tau \omegamq) \bracket{a_{-\vec q} a_{-\vec q}^\dagger} \Theta(-\tau)
    - \exp(- \tau \omegaq) \bracket{a_{\vec q}^\dagger a_{\vec q}} \Theta(-\tau)
    \intertext{%
        The first terms in each row can be simplified by using the commutation
        relation. This will give a occupation number plus 1. This one will stay
        1 in the thermodynamic average. The last terms in each rows give just
        the bosonic occupation number.
    }
    &= - \exp(- \tau \omegaq) [N_{\vec q} + 1] \Theta(\tau)
    - \exp(\tau \omegamq) N_{-\vec q} \Theta(\tau)
    \\&\quad
    - \exp(\tau \omegamq) [N_{-\vec q} + 1] \Theta(-\tau)
    - \exp(- \tau \omegaq) N_{\vec q} \Theta(-\tau)
    \intertext{%
        Now we can group the terms further. Equation~(9) on the problem set
        suggests $\omegaq = \omegamq$. Using that we can join the terms in each
        row.
    }
    &= - \sbr{\exp(- \tau \omegaq) [N_{\vec q} + 1]
    + \exp(\tau \omegaq) N_{\vec q}} \Theta(\tau)
    \\&\quad
    - \sbr{\exp(\tau \omegaq) [N_{\vec q} + 1]
    + \exp(- \tau \omegaq) N_{\vec q}} \Theta(-\tau)
\end{align*}
That is the desired result as given on the problem set.

\begin{question}
    Why does $\omegaq = \omegamq$ hold? Is that some assumption that I missed?
\end{question}

\subsection{Fourier coefficients}

So we now want to compute the Fourier coefficients of Green's function. The
function in imaginary time is given by
\[
    D^{(0)}(\vec q, \tau)
    =
    - \Theta(\tau) \sbr{
        [1 + N_{\vec q} \eup^{-\tau \omegaq} + N_{\vec q} \eup^{\tau \omegaq}
    }
    - \Theta(-\tau) \sbr{
        [1 + N_{\vec q} \eup^{\tau \omegaq} + N_{\vec q} \eup^{-\tau \omegaq}
    }
\]
The Fourier transform is easily compute \emph{if} one recalls that imaginary
time only extends to $\beta$ and then repeats itself \Winkey.

So transform with
positive sign convention.
\begin{align*}
    \tilde D^{(0)}(\vec q, \nu_n)
    &=
    - \int_0^\beta \dif \tau \, \eup^{\iup \tau \nu_n}
    \sbr{
        \Theta(\tau) \sbr{
            [1 + N_{\vec q}] \eup^{-\tau \omegaq} + N_{\vec q} \eup^{\tau \omegaq}
        }
        \Theta(-\tau) \sbr{
            [1 + N_{\vec q}] \eup^{\tau \omegaq} + N_{\vec q} \eup^{-\tau \omegaq}
        }
    }
    \intertext{%
        The $-\tau$ branch can directly be discarded due to the limits of the
        integral.
    }
    &=
    - \int_0^\beta \dif \tau \, \eup^{\iup \tau \nu_n}
    \sbr{
        [1 + N_{\vec q}] \eup^{-\tau \omegaq} + N_{\vec q} \eup^{\tau \omegaq}
    }
    \intertext{%
        Then we can join the exponentials.
    }
    &= - \int_0^\beta \dif \tau
    \sbr{
        [1 + N_{\vec q}] \eup^{\tau[\iup \nu_n - \omegaq]}
        + N_{\vec q} \eup^{\tau [\iup \nu_n + \omegaq]}
    }
    \intertext{%
        The integration itself is not hard, it will give us
    }
    &= -
    \frac{1 + N_{\vec q}}{\iup \nu_n - \omegaq} \sbr{\eup^{\beta[\iup \nu_n -
    \omegaq]} - 1}
    - \frac{N_{\vec q}}{\iup \nu_n + \omegaq} \sbr{\eup^{\beta [\iup \nu_n +
    \omegaq]} - 1}
    \intertext{%
        We use that $\nu_n = 2 \piup n / \beta$. Therefore we can drop that
        drop that from the exponentials.
    }
    &= -
    \frac{1 + N_{\vec q}}{\iup \nu_n - \omegaq} \sbr{ \eup^{-\beta\omegaq} - 1}
    - \frac{N_{\vec q}}{\iup \nu_n + \omegaq} \sbr{\eup^{\beta\omegaq} - 1}
    \intertext{%
        Also note that $1 + N_{\vec q} = \eup^{\beta \omega_n} N_{\vec q}$.
        Therefore we can simplify the expression a bit further.
    }
    &= -
    \frac{N_{\vec q}}{\iup \nu_n - \omegaq} \sbr{1 - \eup^{\beta\omegaq}}
    - \frac{N_{\vec q}}{\iup \nu_n + \omegaq} \sbr{\eup^{\beta\omegaq} - 1}
    \intertext{%
        Now we can cancel $N_{\vec q}$ from both terms. The first will obtain
        another minus sign.
    }
    &= \frac{1}{\iup \nu_n - \omegaq}
    - \frac{1}{\iup \nu_n + \omegaq}
    \intertext{%
        Next we bring the two fractions to equal denominators. We directly
        pull out the minus that we get in the denominator.
    }
    &= - \frac{\iup \nu_n + \omegaq - \iup \nu_n + \omegaq}{\nu_n^2 -
    \omegaq^2}
    \intertext{%
        And the numerator simplifies to the final result:
    }
    &= - \frac{2 \omegaq}{\nu_n^2 - \omegaq^2}
\end{align*}

\begin{question}
    I think I should know that, but why did imaginary time repeat itself? Just
    because there is some imaginary part in an exponential which then cycles in
    the complex plane?
\end{question}

\section{Another frequency summation}
\label{homework:2}

We are asked to show
\[
    \frac 1\beta \sum_{n}
    \tilde G^{(0)} (\vec p, \omega_n)
    \tilde G^{(0)} (\vec k, \omega_n + \nu_m)
    = \frac{n_\text F(\epsilon_{\vec p}) - n_\text F(\epsilon_{\vec k})}
        {\iup \nu_m + \epsilon_{\vec p} - \epsilon_{\vec k}} \,.
\]
I took the liberty to give the $\nu$ another index as this is just one fixed
phonon frequency which we do not sum over.

The overall approach is the same as shown in the tutorial class. The left side
contains a sum over countable infinite many summands. We want to find a
function which has as many residues as the original sum has summands. And the
residues of the poles have to match the summands of the original sum. Then a
contour integral around all the poles will give the sum of the residues. This
then is the sum that we want to compute. The function to be constructed will
also have additional poles. Including all the poles we can stretch the contour
to infinity and have the whole integral vanish. Therefore the finite number of
additional poles will have the same magnitude as all interesting poles
combined. Therefore we can compute the whole sum by computing a few residues
and adding them up.

Now that the direction is clear (hopefully), we can dive into the individual
steps. The left hand side of our equation with explicit form of the Green's
functions is
\[
    \text{LHS} = \frac{1}{\beta} \sum_n \frac{1}{\iup \omega_n - \epsilon_{\vec
    p}} \frac{1}{\iup \omega_n + \iup \nu_n - \epsilon_{\vec k}} \,.
\]
Taking the similar form you did in the tutorial class, we use
\[
    I = \oint \frac{\dif z}{2\piup \iup} \, f(z) \, n_\text F(z)
\]
where
\[
    f(z) = - \frac{1}{z - \epsilon_{\vec p}} \frac{1}{z + \iup \nu_m - \epsilon_{\vec k}}
    \eqnsep
    n_\text F(z) = \frac 12 - \beta \sum_{n = 0}^\infty \sbr{
        \frac1{z + \iup \omega_n}
        +
        \frac1{z - \iup \omega_n}
    } \,.
\]

This construction has poles at $z_1 := \epsilon_{\vec p}$ and $z_2 :=
\epsilon_{\vec k} - \iup \nu_m$ as well as a lot of poles at $z = \pm \iup
\omega_n$. The first two poles lie away from the imaginary axis, the latter all
lie on the imaginary axis. A contour encompassing the imaginary axis will give
the desired result.

The whole function decays with $z^{-2}$ or stronger. The arc length scales with
$z$, so we are good here, the contour integral along a great all-encompassing
circle will vanish. Therefore we know that the sum of all the residues is zero.
Zero minus the residues at $z_1$ and $z_2$ will therefore give the desired
quantity. So all we have to do now is to compute the residues at $z_1$ and
$z_2$. The first one is
\begin{align*}
    R_1
    &= \lim_{z \to z_1} [z - z_1] \, f(z) \, n_\text F(z) \,.
    \intertext{%
        We insert $f(z)$. The other function does not have any poles in this
        region around $z_1$, therefore we do not have to look into that
        function.
    }
    &= - \lim_{z \to \epsilon_{\vec p}} [z - \epsilon_{\vec p}] \frac{1}{z - \epsilon_{\vec p}} \frac{1}{z
    + \iup \nu_m - \epsilon_{\vec k}} \, n_\text F(z) \\
    \intertext{%
        The first fractions conveniently cancels.
    }
    &= - \lim_{z \to \epsilon_{\vec p}} \frac{1}{z + \iup \nu_m -
    \epsilon_{\vec k}} \, n_\text F(z) \\
    \intertext{%
        We can perform the limit now as everything is well behaved.
    }
    &= - \frac{1}{\epsilon_{\vec p} + \iup \nu_m
    - \epsilon_{\vec k}} \, n_\text F(\epsilon_{\vec p}) \\
\end{align*}
Similarly we can compute $R_2$ and obtain
\[
    R_2
    = - \frac{1}{\epsilon_{\vec k} + \iup \nu_m - \epsilon_{\vec p}} \, n_\text
    F(\epsilon_{\vec k} - \iup \nu_m) \,.
\]

The quantity of interest then is
\[
    -R_1 - R_2
    = \frac{n_\text F(\epsilon_{\vec p}) - n_\text F(\epsilon_{\vec k} - \iup
    \nu_m)}{\epsilon_{\vec p} - \epsilon_{\vec k} + \iup \nu_m} \,.
\]
Somehow the second Fermi function is shifted by the phonon frequency. Other
than that it matches the desired right hand side of Equation~(11) given on the
problem set.

\section[Benzene]{A representation of the low-energy physics in the Benzene molecule}
\label{homework:3}

Either I am missing something fundamental or this is a rather easy task. In the
tight binding model we have some energy~$\epsilon$ for the electrons at each site. Then
there might be a next-neighbor hopping amplitude~$t$. The states $\ket i$ are
the orbitals at each lattice site. Then the ladder operators $c_i$ and
$c_i^\dagger$ create an annihilate electrons at the lattice sites.

Defining $c_7 := c_1$ for the periodic boundary conditions, the tight binding
Hamiltonian is
\[
    H = \epsilon \sum_i c_i^\dagger c_i + t \sum_i \sbr{c_i^\dagger c_{i+1} +
    \hc} \,.
\]
In matrix form it can be written as
\[
    H = \vec c^\dagger
    \begin{pmatrix}
        \epsilon & t &&&& \epsilon \\
        t & \epsilon & t &&& \\
          & t & \epsilon & t && \\
          && t & \epsilon & t & \\
          &&& t & \epsilon & t  \\
        \epsilon &&&& t & \epsilon
    \end{pmatrix}
    \vec c
\]
where $\vec c$ is the vector with all the ladder operators.

As a next step this symmetric matrix can be diagonalized by expressing the
ladder operators in momentum space via a Fourier transform. The eigenvalues of
the matrix correspond to the energy bands of the system. One should get
something with cones over the lattice sites if I recall that correctly.

\end{document}

% vim: spell spelllang=en tw=79
