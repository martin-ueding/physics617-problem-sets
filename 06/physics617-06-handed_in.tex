\documentclass[11pt, english, fleqn, DIV=15, headinclude, BCOR=1cm]{scrartcl}

\usepackage[bibatend]{../header}

\usepackage{lastpage}
\usepackage{multicol}
\usepackage{simplewick}
\usepackage{slashed}
\usepackage{subcaption}

\usepackage{../my-boxes}

\newcommand\timeorder{\mathscr T}
\newcommand\normorder{\mathscr N}
\newcommand\eye{\mat 1_4}
\newcommand\myslash[1]{\underline{\slashed{\vec{#1}}}}

\hypersetup{
    pdftitle=
}

\graphicspath{{build/}}

\newcounter{totalpoints}
\newcommand\punkte[1]{#1\addtocounter{totalpoints}{#1}}

\newcounter{problemset}
\setcounter{problemset}{6}

\subject{physics617 -- Theoretical Condensed Matter Physics}
\ihead{physics617 -- Problem Set \arabic{problemset}}

\title{Problem Set \arabic{problemset}}

\newcommand\thegroup{Tutor: Ramsés Sánchez}

\publishers{\thegroup}
\ofoot{\thegroup}

\author{
    Martin Ueding \\ \small{\href{mailto:mu@martin-ueding.de}{mu@martin-ueding.de}}
}
\ifoot{Martin Ueding}

\ohead{\rightmark}

\begin{document}

\maketitle

\vspace{3ex}

\begin{center}
    \begin{tabular}{rrr}
        \toprule
        Problem & Achieved points & Possible points \\
        \midrule
        \nameref{homework:1} & & \punkte{15} \\
        \nameref{homework:2} & & \punkte{10} \\
        \nameref{homework:3} & & \punkte{0} \\
        \midrule
        Total & & \arabic{totalpoints} \\
        \bottomrule
    \end{tabular}
\end{center}

\vspace{3ex}

\begin{center}
    \begin{small}
        This document consists of \pageref{LastPage} pages.
    \end{small}
\end{center}

\section{Free-phonon Green's function}
\label{homework:1}

\subsection{Ladder operators}

\newcommand\omegaq{\omega_{\vec q}}
\newcommand\omegamq{\omega_{-\vec q}}

We have
\[
    a_{\vec q}^\dagger = \eup^{\tau H} a_{\vec q}^\dagger \eup^{-\tau H} \,.
\]
To simplify this, we want to use the Baker-Campbell-Hausdorff formula with $A =
\tau H$ and $B = a_{\vec q}^\dagger$. The commutator of $A$ and $B$ is
\begin{align*}
    [A, B]
    &= [\tau H, a_{\vec q}^\dagger] \,.
    \intertext{%
        We insert the explicit form of the Hamiltonian.
    }
    &= \tau \sum_{\vec k} \omega_{\vec k}
    [a_{\vec k}^\dagger a_{\vec k}, a_{\vec q}^\dagger]
    \intertext{%
        Since the raising operator always commutes with itself, we can pull
        that out of the commutator already.
    }
    &= \tau \sum_{\vec k} \omega_{\vec k}
    a_{\vec k}^\dagger [a_{\vec k}, a_{\vec q}^\dagger]
    \intertext{%
        This commutator is just $\delta(\vec k - \vec q)$.
    }
    &= \tau \omegaq a_{\vec q}^\dagger
\end{align*}

We see that the commutator $[A, B]$ is just $\tau \omegaq B$ again. Therefore
the next commutator is $[A, [A, B]] = \tau\omegaq [A, B]$. We can factor out
the remaining $B$ and yield
$\exp(\tau \omegaq) B$ from all the terms. Our result therefore is
\[
    a_{\vec q}^\dagger(\tau) = \exp(\tau \omegaq) a_{\vec q}^\dagger \,.
\]

The derivation for $a_{\vec q}$ is very similar. The sign in the exponent
changes as the two terms in the commutator are now in the wrong order, they
need to be switched first before the canonical commutation relation can be
applied. Therefore we have
\[
    a_{\vec q}(\tau) = \exp(- \tau \omegaq) a_{\vec q} \,.
\]

\subsection{Green's function}

First we have to build up $A(\vec q, \tau)$ from the time dependent ladder
operators that we have just derived. Given is
\begin{align*}
    A(\vec q, \tau)
    &= \eup^{\tau H} [a_{\vec q} + a_{- \vec q}^\dagger] \eup^{- \tau H} \,.
    \intertext{%
        There we insert the expressions for the ladder operators.
    }
    &= \exp(- \tau \omegaq) a_{\vec q} + \exp(\tau \omegamq) a_{-\vec q}^\dagger
\end{align*}

Having that, we can compute the Green's function. The definition is
\begin{align*}
    D(\vec q, \tau)
    &= - \bracket{T_\tau A(\vec q, t) A(- \vec q, 0)} \,.
    \intertext{%
        Then we insert $A$.
    }
    &= - \bracket{T_\tau
    \sbr{\exp(- \tau \omegaq) a_{\vec q} + \exp(\tau \omegamq) a_{-\vec q}^\dagger}
    [a_{-\vec q} + a_{\vec q}^\dagger]}
    \intertext{%
        We expand the time ordering.
    }
    &= - \bracket{\sbr{\exp(- \tau \omegaq) a_{\vec q} + \exp(\tau \omegamq) a_{-\vec q}^\dagger}
    [a_{-\vec q} + a_{\vec q}^\dagger]} \Theta(\tau)
    \\&\quad
    - \bracket{[a_{-\vec q} + a_{\vec q}^\dagger]
    \sbr{\exp(- \tau \omegaq) a_{\vec q} + \exp(\tau \omegamq) a_{-\vec
    q}^\dagger}} \Theta(-\tau)
    \intertext{%
        Only number operators contribute to the traces. Therefore only two
        products for each time branch do anything.
    }
    &= - \bracket{\exp(- \tau \omegaq) a_{\vec q} a_{\vec q}^\dagger
    + \exp(\tau \omegamq) a_{-\vec q}^\dagger a_{-\vec q}} \Theta(\tau)
    \\&\quad
    - \bracket{\exp(\tau \omegamq) a_{-\vec q} a_{-\vec q}^\dagger
    + \exp(- \tau \omegaq) a_{\vec q}^\dagger a_{\vec q}} \Theta(-\tau)
    \intertext{%
        The exponentials can be taken out of the traces.
    }
    &= - \exp(- \tau \omegaq) \bracket{a_{\vec q} a_{\vec q}^\dagger} \Theta(\tau)
    - \exp(\tau \omegamq) \bracket{a_{-\vec q}^\dagger a_{-\vec q}} \Theta(\tau)
    \\&\quad
    - \exp(\tau \omegamq) \bracket{a_{-\vec q} a_{-\vec q}^\dagger} \Theta(-\tau)
    - \exp(- \tau \omegaq) \bracket{a_{\vec q}^\dagger a_{\vec q}} \Theta(-\tau)
    \intertext{%
        The first terms in each row can be simplified by using the commutation
        relation. This will give a occupation number plus 1. This one will stay
        1 in the thermodynamic average. The last terms in each rows give just
        the bosonic occupation number.
    }
    &= - \exp(- \tau \omegaq) [N_{\vec q} + 1] \Theta(\tau)
    - \exp(\tau \omegamq) N_{-\vec q} \Theta(\tau)
    \\&\quad
    - \exp(\tau \omegamq) [N_{-\vec q} + 1] \Theta(-\tau)
    - \exp(- \tau \omegaq) N_{\vec q} \Theta(-\tau)
    \intertext{%
        Now we can group the terms further. Equation~(9) on the problem set
        suggests $\omegaq = \omegamq$. Using that we can join the terms in each
        row.
    }
    &= - \sbr{\exp(- \tau \omegaq) [N_{\vec q} + 1]
    + \exp(\tau \omegaq) N_{\vec q}} \Theta(\tau)
    \\&\quad
    - \sbr{\exp(\tau \omegaq) [N_{\vec q} + 1]
    + \exp(- \tau \omegaq) N_{\vec q}} \Theta(-\tau)
\end{align*}
That is the desired result as given on the problem set.

\begin{question}
    Why does $\omegaq = \omegamq$ hold? Is that some assumption that I missed?
\end{question}

\subsection{Fourier coefficients}

\section{Another frequency summation}
\label{homework:2}

\section[Benzene]{A representation of the low-energy physics in the Benzene molecule}
\label{homework:3}

\end{document}

% vim: spell spelllang=en tw=79
