\documentclass[11pt, english, fleqn, DIV=15, headinclude, BCOR=1cm]{scrartcl}

\usepackage[bibatend]{../header}

\usepackage{lastpage}
\usepackage{multicol}
\usepackage{simplewick}
\usepackage{slashed}
\usepackage{subcaption}
\usepackage{minted}

\newcommand\timeorder{\mathscr T}
\newcommand\normorder{\mathscr N}
\newcommand\eye{\mat 1_4}
\newcommand\myslash[1]{\underline{\slashed{\vec{#1}}}}

\hypersetup{
    pdftitle=
}

\graphicspath{{build/}}

\newcounter{totalpoints}
\newcommand\punkte[1]{#1\addtocounter{totalpoints}{#1}}

\newcounter{problemset}
\setcounter{problemset}{4}

\subject{physics617 -- Theoretical Condensed Matter Physics}
\ihead{physics617 -- Problem Set \arabic{problemset}}

\title{Problem Set \arabic{problemset}}

\newcommand\thegroup{Tutor: Ramsés Sánchez}

\publishers{\thegroup}
\ofoot{\thegroup}

\author{
    Martin Ueding \\ \small{\href{mailto:mu@martin-ueding.de}{mu@martin-ueding.de}}
}
\ifoot{Martin Ueding}

\ohead{\rightmark}

\begin{document}

\maketitle

\vspace{3ex}

\begin{center}
    \begin{tabular}{rrr}
        \toprule
        Problem & Achieved points & Possible points \\
        \midrule
        \nameref{homework:1} & & \\
        \nameref{homework:2} & & \\
        \midrule
        %Total & & \arabic{totalpoints} \\
        Total & & 25 \\
        \bottomrule
    \end{tabular}
\end{center}

\vspace{3ex}

\begin{center}
    \begin{small}
        This document consists of \pageref{LastPage} pages.
    \end{small}
\end{center}

\section{Density of states of a phonon mode of a 2D system}
\label{homework:3}

\subsection{Numerical computation}

A dispersion
\[
    E(\vec k) \propto \sqrt{2 - \cos(k_x) - \cos(k_y)}
\]
is given. The energy values lie in the interval $[0, 2]$. The wave vectors
$\vec k$ lie in the first Brillouin zone which is $(-\piup, \piup]^2$ here. Due
to the symmetry of the dispersion relation it suffices to iterate in the
interval $[0, \piup]^2$.

We haven chosen the approach to sample a grid in the interval $[0, \piup]^2$
and compute the energy for each of those wave vectors. Then we build up a
histogram with the energies. The resulting distribution is the density of
states with respect to the energy.

Listing~\ref{lst:cpp} shows the C++ version of it. Except for the use of the
brace initialization this would compile with the C++03 standard. It uses a
simple histogram implementation which is enough for this application. After the
sampling of the grid it will scale the histogram counts such that the end
result is a proper density function with unit integral. Listing~\ref{lst:cmake}
shows the short CMake file to compile the program.

\begin{listing}[tb]
    \inputminted[linenos, fontsize=\footnotesize]{cpp}{dos.cpp}
    \caption{%
        C++ program for density of state computation.
    }
    \label{lst:cpp}
\end{listing}

\begin{listing}[tb]
    \inputminted[linenos, fontsize=\footnotesize]{cmake}{CMakeLists.txt}
    \caption{%
        CMake build file for C++ program.
    }
    \label{lst:cmake}
\end{listing}

We implemented the same algorithm again in Python with the NumPy and SciPy
libraries. This is shown in Listing~\ref{lst:python}. In order to benefit from
the native implementation of the NumPy routines we have chosen not to iterate
the grid with Python itself but using 2D arrays of the grid points. As a last
step we compute the integral of the density function to make sure that the
integral is approximately unity.

\begin{listing}[tb]
    \inputminted[linenos, fontsize=\footnotesize]{python}{dos.py}
    \caption{%
        Python program for density of state computation.
    }
    \label{lst:python}
\end{listing}

The results of both programs are shown in Figure~\ref{fig:dos-numerical}. The
C++ version was run with a lot more sampling points and more bins. Therefore it
is not surprising that the result looks better.

\begin{figure}
    \centering
    \includegraphics{dos-numerical}
    \caption{%
        Numerical density of states function.
    }
    \label{fig:dos-numerical}
\end{figure}


\end{document}

% vim: spell spelllang=en tw=79
