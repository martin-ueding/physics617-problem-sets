\documentclass[11pt, english, fleqn, DIV=15, headinclude, BCOR=1cm]{scrartcl}

\usepackage[bibatend]{../header}

\usepackage{lastpage}
\usepackage{multicol}
\usepackage{simplewick}
\usepackage{slashed}
\usepackage{subcaption}

\newcommand\timeorder{\mathscr T}
\newcommand\normorder{\mathscr N}
\newcommand\eye{\mat 1_4}
\newcommand\myslash[1]{\underline{\slashed{\vec{#1}}}}

\hypersetup{
    pdftitle=
}

\graphicspath{{build/}}

\newcounter{totalpoints}
\newcommand\punkte[1]{#1\addtocounter{totalpoints}{#1}}

\newcounter{problemset}
\setcounter{problemset}{2}

\subject{physics617 -- Theoretical Condensed Matter Physics}
\ihead{physics617 -- Problem Set \arabic{problemset}}

\title{Problem Set \arabic{problemset}}

\newcommand\thegroup{Tutor: Ramsés Sánchez}

\publishers{\thegroup}
\ofoot{\thegroup}

\author{
    Martin Ueding \\ \small{\href{mailto:mu@martin-ueding.de}{mu@martin-ueding.de}}
}
\ifoot{Martin Ueding}

\ohead{\rightmark}

\begin{document}

\maketitle

\vspace{3ex}

\begin{center}
    \begin{tabular}{rrr}
        \toprule
        Problem & Achieved points & Possible points \\
        \midrule
        \nameref{homework:1} & & \punkte{25} \\
        %\midrule
        %Total & & \arabic{totalpoints} \\
        \bottomrule
    \end{tabular}
\end{center}

\vspace{3ex}

\begin{center}
    \begin{small}
        This document consists of \pageref{LastPage} pages.
    \end{small}
\end{center}

\section{Atoms as Dirac delta potentials}
\label{homework:1}

\subsection{Jump in derivative}

The potential here is $V(x) = V_0 \deltaup(x)$. The corresponding
time-independent Schrödinger equation then is:
\[
    \sbr{- \frac{1}{2m} \laplace + V_0 \deltaup(x) - E} \psi(x) = 0 \,.
\]
In order to find the jump at the position of the atoms, we integrate around a
$2 \epsilon$ ($\epsilon \in \R^+$) sized region centered around $x = 0$.
\begin{align*}
    \int_{-\epsilon}^\epsilon \dif x \, \sbr{- \frac{1}{2m} \laplace + V_0 \deltaup(x) - E} \psi(x) &= 0
    \intertext{%
        We split up the integral in multiple parts.
    }
    - \frac{1}{2m} \int_{-\epsilon}^\epsilon \dif x \, \psi''(x) + V_0 \int_{-\epsilon}^\epsilon \dif x \, \deltaup(x) \psi(x) - \int_{-\epsilon}^\epsilon \dif x \, E \psi(x) &= 0
    \intertext{%
        The first integral can be simplified with Stokes's theorem. The second
        is just the evaluation of the function at $x = 0$. The last integral
        will give a contribution in the order of $\epsilon$.
    }
    - \frac{1}{2m} \sbr{\psi'(\epsilon) - \psi'(-\epsilon)}
    + V_0 \psi(0)
    + \mathrm O(\epsilon)
    &= 0
\end{align*}
We isolate the jump in the derivative.
\[
    \sbr{\psi'(\epsilon) - \psi'(-\epsilon)}
    = 2m V_0 \psi(0) + \mathrm O(\epsilon)
\]
Then we can take the limit and obtain the actual jump.
\[
    \lim_{\epsilon \to 0}
    \sbr{\psi'(\epsilon) - \psi'(-\epsilon)}
    = 2m V_0 \psi(0)
    = 2m V_0 [A + B]
\]
There is a jump in the case that $A + B \neq 0$. The case $A = B = 0$ is not
interesting as it is a state that could not even be normalized. The state $A =
-B$ would make $\psi(x) \propto \sin(Kx)$ which will avoid the potential peak
at $x = 0$ and is not affected by it.

\subsection{Extension to the full lattice}

We know that the wave function in the free part between the atoms is given by
Equation~(4) from the problem set:
\[
    \psi(x) = A \exp(\iup Kx) + B \exp(- \iup Kx) \,.
\]
On the left and on the right side of the atom located at $x = 0$ we can write
the wave function as $\psi$ as glue them together with Heaviside step
functions.
\[
    \psi(x) = \Theta(-x) \sbr{A \exp(\iup Kx) + B \exp(- \iup Kx)}
    + \Theta(x) \sbr{C \exp(\iup Kx) + D \exp(- \iup Kx)} \,.
\]
The relation of those coefficients is going to be of interest. From the
continuity of the wave function over the atoms we know that
\[
    \lim_{\epsilon \to 0} \sbr{\psi(\epsilon) - \psi(-\epsilon)} = 0 \,.
\]
Inserting the combined wave function this will lead to the equation
\[
    A + B - C - D = 0
\]
which tells us $A+B$ is the same as $C+D$. That will come in handy along the
way.

The Schrödinger equation contains the second derivative of the wave function.
Therefore we must differentiate it twice. We start with the first derivative.
\begin{align*}
    \psi'(x)
    &= - \delta(-x) \sbr{A \exp(\iup Kx) + B \exp(- \iup Kx)}
    + \Theta(-x) \iup k \sbr{A \exp(\iup Kx) - B \exp(- \iup Kx)}
    \\&\quad
    + \delta(x) \sbr{C \exp(\iup Kx) + D \exp(- \iup Kx)}
    + \Theta(x) \iup k \sbr{C \exp(\iup Kx) - D \exp(- \iup Kx)}
    \intertext{%
        Here we have used that the Dirac $\delta$-distribution is the
        derivative of the Heaviside step function~$\Theta$. The additional
        minus sign come from the chain rule. Next we evaluate the square
        bracket after the $\delta$ at $x = 0$ since that is a continuous
        function and would happen anyway when we integrate over the whole
        expression.
    }
    &= - \delta(-x) [A + B]
    + \delta(x) [C + D]
    \\&\quad
    + \Theta(-x) \iup k \sbr{A \exp(\iup Kx) - B \exp(- \iup Kx)}
    + \Theta(x) \iup k \sbr{C \exp(\iup Kx) - D \exp(- \iup Kx)}
    \intertext{%
        We can replace $C + D$ with $A + B$ and see that the first two summands
        cancel.
    }
    &= \Theta(-x) \iup k \sbr{A \exp(\iup Kx) - B \exp(- \iup Kx)}
    + \Theta(x) \iup k \sbr{C \exp(\iup Kx) - D \exp(- \iup Kx)}
\end{align*}
So that was the first derivative. We will differentiate again.
\begin{align*}
    \psi''(x)
    &= 
    - \delta(-x) \iup k \sbr{A \exp(\iup Kx) - B \exp(- \iup Kx)}
    - \Theta(-x) k^2 \sbr{A \exp(\iup Kx) + B \exp(- \iup Kx)}
    \\&\quad
    - \delta(x) \iup k \sbr{C \exp(\iup Kx) - D \exp(- \iup Kx)}
    - \Theta(x) k^2 \sbr{C \exp(\iup Kx) + D \exp(- \iup Kx)}
    \intertext{%
        Again we can do the same thing with the $\delta$-distribution. The two
        summands with the Heaviside function are just the wave function $\psi$
        again.
    }
    &= - \delta(-x) \iup k [A - B]
    - \delta(x) \iup k [C - B]
    - k^2 \psi(x)
    \intertext{%
        The first two terms can be combined. We obtain the final result for the
        second derivative:
    }
    &= \delta(x) \iup k [- A + B + C - D] - k^2 \psi(x) \,.
\end{align*}

At this point we have all the parts to assemble the Schrödinger equation.
\begin{align*}
    0
    &= \sbr{- \frac{\laplace}{2m} + V - E} \psi(x)
    \intertext{%
        Inserting the terms gives us
    }
    &= \frac{\iup k}{2m} \delta(x) [A - B - C + D]
    + \sbr{\frac{k^2}{2m} + V_0 \delta(x) - E} \psi(x) \,.
\end{align*}

In case $x \neq 0$ we should have
\[
    E = \frac{K^2}{2m}
\]
which is indeed the definition of $K$. For the case $x = 0$ it would be
convenient to have the other terms vanish independent of $\psi$.
\begin{align*}
    0 &= \delta(x) \sbr{\frac{\iup k}{2m} [A - B - C + D] + V_0 \psi(x)}
    \intertext{%
        With the $\delta(x)$ there, we can directly write $A + B$ (or $C + D$)
        for $\psi(x)$ as we did in the derivatives as well.
    }
    &= \delta(x) \sbr{\frac{\iup k}{2m} [A - B - C + D] + V_0 [A + B]}
    \intertext{%
        This equation is certainly satisfied if the square bracket vanishes, so
        we will see how we can do this.
    }
    \impliedby 0 &= \frac{\iup k}{2m} [A - B - C + D] + V_0 [A + B]
    \intertext{%
        We will replace $D$ with $A + B - C$.
    }
    &= \frac{\iup k}{m} [A - C] + V_0 [A + B]
\end{align*}

We would like to have two functions, namely $A(C, D)$ and $B(C, D)$. Those can
be acquired by massaging the above equation into a suitable form. Together with
$A + B = C + D$ we obtain
\[
    A = \frac{\iup m V_0}{K} [C + D] + C
    \eqnsep
    B = D - \frac{\iup m V_0}{K} [C + D] \,.
\]
This can be written as a matrix equation:
\[
    \begin{pmatrix}
        A \\ B
    \end{pmatrix}
    =
    \begin{pmatrix}
        \frac{\iup m V_0}{K} + 1 & \frac{\iup m V_0}{K} \\
        - \frac{\iup m V_0}{K} & 1 - \frac{\iup m V_0}{K}
    \end{pmatrix}
    \begin{pmatrix}
        C \\ D
    \end{pmatrix}
    \,.
\]
In order to remove the visual clutter we define
\[
    c := \frac{\iup m V_0}{K}
\]
and use that in the matrix equation:
\[
    \begin{pmatrix}
        A \\ B
    \end{pmatrix}
    =
    \begin{pmatrix}
        c + 1 & c \\
        - c & 1 - c
    \end{pmatrix}
    \begin{pmatrix}
        C \\ D
    \end{pmatrix}
    \,.
\]
This equation allows us to carry coefficients from one cell to another cell on
the left (smaller $x$).

The idea now is to translate the function $u_k(x)$ by the lattice spacing $a$
to the right (larger $x$). Then the coefficients $A$ and $B$ change but the
function should stay the same. The new coefficients can be “pulled back” using
the matrix equation. This operation should not change the coefficients. We will
express the shifting of $u_k(x)$ as a matrix which acts on the coefficients as
well. The product of both matrices needs to have an eigenvalue of unity such
that the periodicity actually is preserved. The needed translation matrix for
$u_k(x)$ is
\[
    \begin{pmatrix}
        \exp(\iup [K - k] a) & 0 \\
        0 & \exp(\iup [-K - k] a)
    \end{pmatrix} \,.
\]
The matrix product which does one round-trip is
\[
    \begin{pmatrix}
        [c+1] \exp(\iup [K - k] a) & c \exp(\iup [- K - k] a) \\
        -c \exp(\iup [K - k] a) & [1-c] \exp(\iup [- K - k] a) \\
    \end{pmatrix} \,.
\]
This one has to have an eigenvalue of unity. We subtract $\mat 1_2$ from this
and set the determinant equal to zero. This gives us the following equation:
\begin{align*}
    0
    &= \cbr{[c+1] \exp(\iup [K - k] a) - 1} \cbr{[1-c] \exp(\iup [- K - k] a) - 1}
    \\&\quad
    + c^2 \exp(\iup [- K - k] a) \exp(\iup [K - k] a) \,.
    \intertext{%
        In the first term, we need to expand this into four terms. We can
        combine the exponentials in the last term.
    }
    &= [c+1] [1-c] \exp(\iup [K - k] a) \exp(\iup [- K - k] a)
    - [c+1] \exp(\iup [K - k] a)
    \\&\quad
    - [1-c] \exp(\iup [- K - k] a)
    + 1
    + c^2 \exp(- 2 \iup ka)
    \intertext{%
        We simplify the first term a bit.
    }
    &= [1 - c^2] \exp(- 2 \iup k a) 
    - [c+1] \exp(\iup [K - k] a)
    - [1-c] \exp(\iup [- K - k] a)
    + 1
    \\&\quad
    + c^2 \exp(- 2 \iup ka)
    \intertext{%
        The second term in the first bracket (the $-c^2$ term) and the very
        last one cancel. We move the $+1$ term to the second position, this
        will become clear in a bit.
    }
    &= \exp(- 2 \iup k a) 
    + 1
    - [c+1] \exp(\iup [K - k] a)
    - [1-c] \exp(\iup [- K - k] a)
    \intertext{%
        From the exponentials, we can factor out the term going with $k$. From
        the $+1$ term we can do this as well.
    }
    &= \exp(- \iup k a) \exp(- \iup k a) 
    + \exp(- \iup k a) \exp(\iup k a) 
    \\&\quad
    - \exp(- \iup k a) \cbr{[c+1] \exp(\iup K a) + [1-c] \exp(- \iup K a)}
    \intertext{%
        Now all the terms have an exponential up front, we factor that out.
        Then we drop it as it cannot be zero and the remainder needs to be
        zero.
    }
    \impliedby 0 &= \exp(- \iup k a) 
    + \exp(\iup k a) 
    - [c+1] \exp(\iup K a) - [1-c] \exp(- \iup K a)
    \intertext{%
        The first two terms form a cosine. For the last two terms we need to
        expand the square bracket in front first.
    }
    &= 2 \cos(ka)
    - \exp(\iup K a) - \exp(- \iup K a)
    - c \exp(\iup K a) + c \exp(- \iup K a)
    \intertext{%
        Then we have another cosine and another sine here.
    }
    &= 2 \cos(ka) - 2 \cos(Ka) - 2 \iup c \sin(Ka) \\
    \intertext{%
        We cancel the factor of two and insert $c$.
    }
    0 &= \cos(ka) - \cos(Ka) - \iup \frac{i m V_0}{K} \sin(Ka)
    \intertext{%
        And this gives us the desired result:
    }
    0 &= \cos(ka) - \cos(Ka) + \frac{m V_0}{K} \sin(Ka) \,.
\end{align*}
By comparing the coefficients we gather that $\alpha = - m V_0$. If the
potential was defined as being negative, this sign might be the other way
around. On my problem set, the potential is defined as being $V(x) = V_0
\delta(x)$.

\end{document}

% vim: spell spelllang=en tw=79
