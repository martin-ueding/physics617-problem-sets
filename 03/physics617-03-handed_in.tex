\documentclass[11pt, english, fleqn, DIV=15, headinclude, BCOR=1cm]{scrartcl}

\usepackage[bibatend]{../header}

\usepackage{lastpage}
\usepackage{multicol}
\usepackage{simplewick}
\usepackage{slashed}
\usepackage{subcaption}

\newcommand\timeorder{\mathscr T}
\newcommand\normorder{\mathscr N}
\newcommand\eye{\mat 1_4}
\newcommand\myslash[1]{\underline{\slashed{\vec{#1}}}}

\hypersetup{
    pdftitle=
}

\graphicspath{{build/}}

\newcounter{totalpoints}
\newcommand\punkte[1]{#1\addtocounter{totalpoints}{#1}}

\newcounter{problemset}
\setcounter{problemset}{2}

\subject{physics617 -- Theoretical Condensed Matter Physics}
\ihead{physics617 -- Problem Set \arabic{problemset}}

\title{Problem Set \arabic{problemset}}

\newcommand\thegroup{Tutor: Ramsés Sánchez}

\publishers{\thegroup}
\ofoot{\thegroup}

\author{
    Martin Ueding \\ \small{\href{mailto:mu@martin-ueding.de}{mu@martin-ueding.de}}
}
\ifoot{Martin Ueding}

\ohead{\rightmark}

\begin{document}

\maketitle

\vspace{3ex}

\begin{center}
    \begin{tabular}{rrr}
        \toprule
        Problem & Achieved points & Possible points \\
        \midrule
        \nameref{homework:1} & & \\
        \nameref{homework:2} & & \\
        \midrule
        %Total & & \arabic{totalpoints} \\
        Total & & 25 \\
        \bottomrule
    \end{tabular}
\end{center}

\vspace{3ex}

\begin{center}
    \begin{small}
        This document consists of \pageref{LastPage} pages.
    \end{small}
\end{center}

\section{Linear chain with next-nearest neighbor hopping}
\label{homework:1}

\subsection{Dispersion}

We are given a linear chain with distance~$a$ between the sites. The there
first- and second-neighbor hopping amplitudes $t_1$ and $t_2$. The situation is
depicted in Figure~\ref{fig:hopping}.

\begin{figure}[tb]
    \centering
    \includegraphics{hopping}
    \caption{%
        Hopping amplitudes between the lattice sites.
    }
    \label{fig:hopping}
\end{figure}

Instead of building up a Hamiltonian from (unknown) potential and kinetic
energy we will just use the hopping amplitudes as was done in the in-class
exercise “Tight-binding model on a honeycomb lattice”. The Hamiltonian then
looks like this:
\[
    H =
    - t_1 \sum_{i, j \; \text{nearest neighbors}}
    [c_i^\dagger c_j + \hc]
    - t_2 \sum_{i, j \; \text{next nearest neighbors}}
    [c_i^\dagger c_j + \hc] \,.
\]
There is probably nice notation like $\braket{i, j}$ for the first part. I do
not know how to deal with the second part, so we will just use indices for
that. Then the Hamiltonian will take take the following form:
\[
    H =
    - \sum_i
    [c_i^\dagger c_{j+1} + c_i^\dagger c_{j+2} + \hc] \,.
\]
In total there are four terms which correspond to the four possible hopping
distances of \numlist{+1;+2;-1;-2}. This Hamiltonian has a similar form
compared to the one shown on Monday in class. It is not diagonal and needs to
be diagonalized to give sensible energy eigenvalues. The diagonalization starts
with a Fourier transformation of the ladder operators. It is given as
\[
    c_j = \frac{1}{\sqrt{N}} \sum_k c_k \exp(\iup kaj) \,,
\]
where $N$ is the number of lattice sites, $j$ the lattice site, $k$ is a
momentum and $a$ the spacing of the lattice. Although the imaginary unit~$\iup$
and the index~$i$ can distinguished in print, we try not to mix them too much.




\begin{figure}[tb]
    \begin{subfigure}[c]{0.5\linewidth}
        \centering
        \includegraphics{linear-band}
        \caption{%
            $t_1 = 1$, $t_2 = 1$
        }
        \label{fig:linear-band/1}
    \end{subfigure}
    \begin{subfigure}[c]{0.5\linewidth}
        \centering
        \includegraphics{linear-band-0_5__-0_5}
        \caption{%
            $t_1 = 1/2$, $t_2 = -1/2$
        }
        \label{fig:linear-band/2}
    \end{subfigure}
    \caption{%
        Band structure of the linear chain. The bold line is the solution where
        the square root is added (instead of subtracted).
    }
    \label{fig:linear-band}
\end{figure}

\subsection{Number of minima}

\subsection{Density of states}

I must admit that I was thinking of few other things before I realized that
“DOS” is supposed to stand for:

\begin{itemize}
    \item Denial of Service
    \item Degrees of Separation
    \item Disk Operating System
\end{itemize}

\section{Tight-binding model on a kagomé lattice}
\label{homework:2}


\end{document}

% vim: spell spelllang=en tw=79
