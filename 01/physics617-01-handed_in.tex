\documentclass[11pt, english, fleqn, DIV=15, headinclude, BCOR=1cm]{scrartcl}

\usepackage[bibatend]{../header}

\usepackage{lastpage}
\usepackage{multicol}
\usepackage{simplewick}
\usepackage{slashed}
\usepackage{subcaption}

\newcommand\timeorder{\mathscr T}
\newcommand\normorder{\mathscr N}
\newcommand\eye{\mat 1_4}
\newcommand\myslash[1]{\underline{\slashed{\vec{#1}}}}

\hypersetup{
    pdftitle=
}

\graphicspath{{build/}}

\newcounter{totalpoints}
\newcommand\punkte[1]{#1\addtocounter{totalpoints}{#1}}

\newcounter{problemset}
\setcounter{problemset}{1}

\subject{physics617 -- Theoretical Condensed Matter Physics}
\ihead{physics617 -- Problem Set \arabic{problemset}}

\title{Problem Set \arabic{problemset}}

\newcommand\thegroup{Tutor: Ramsés Sánchez}

\publishers{\thegroup}
\ofoot{\thegroup}

\author{
    Martin Ueding \\ \small{\href{mailto:mu@martin-ueding.de}{mu@martin-ueding.de}}
}
\ifoot{Martin Ueding}

\ohead{\rightmark}

\begin{document}

\maketitle

\vspace{3ex}

\begin{center}
    \begin{tabular}{rrr}
        Problem & Achieved points & Possible points \\
        \midrule
        \nameref{homework:1} & & \punkte{5} \\
        \nameref{homework:2} & & \punkte{20} \\
        \midrule
        Total & & \arabic{totalpoints}
    \end{tabular}
\end{center}

\vspace{3ex}

\begin{center}
    \begin{small}
        This document consists of \pageref{LastPage} pages.
    \end{small}
\end{center}

\vspace{3ex}

I would like to scan and upload the problem sets with your corrections to my
website \href{http://martin-ueding.de}{martin-ueding.de}. There, the original
problem set as well as the reviewed one will be licensed under the
“\href{http://creativecommons.org/licenses/by-sa/4.0/}{Creative Commons
Attribution-ShareAlike 4.0 International License}”. Is that okay with you?

Yes $\Box$ \hspace{2cm} No $\Box$

\section{Sodium in hcp und bcc}
\label{homework:1}

We have a transition from bcc to hcp with a fixed density. This density is the
number of atoms per volume. The volume is best taken to be the unit cell. The
bcc-lattice is shown in Figure~\ref{fig:bcc}. There are eight atoms on the
corners of the cell. Only one octant contributes, so in total that is one atom.
The atom in the middle also contributes, therefore there are two atoms in the
bcc unit cell.

\begin{figure}[htbp]
    \centering
    \includegraphics{bcc}
    \caption{%
        bcc-unit cell. Atom “suspended” by dashed lines is in the center of the
        cube. Not all atoms shown actually belong to the unit cell.
    }
    \label{fig:bcc}
\end{figure}

The density therefore is:
\[
    \rho = \frac{2}{V}
    \quad\text{with}\quad
    V = a^3 \,.
\]

The unit cell of the hcp-lattice is shown in Figure~\ref{fig:hcp-unit}. There
are two atoms per unit cell as well. The Bravais vectors of this system are the
following:
\[
    \tilde{\vec a}_1 =
    \tilde a
    \begin{pmatrix}
        1 \\ 0 \\ 0
    \end{pmatrix}
    \quad
    \tilde{\vec a}_2 =
    \tilde a
    \begin{pmatrix}
        1/2 \\ \sqrt{3}/2 \\ 0
    \end{pmatrix}
    \quad
    \tilde{\vec a}_3 =
    \tilde a
    \begin{pmatrix}
        0 \\ 0 \\ \sqrt{8/3}
    \end{pmatrix}
\]

\begin{figure}[htbp]
    \centering
    \includegraphics{hcp-unit}
    \caption{%
        Unit cell of hcp-lattice. The three Bravais vectors are shown as well
        as the atoms belonging to the unit cell. The atom “suspended” by the
        dashed lines actually sits above the middle of the triangle
        ($\bigtriangleup$) suggested by the dotted line. It seems hard to find
        a decent projection for those lattices.
    }
    \label{fig:hcp-unit}
\end{figure}

The volume of this cell is
\[
    \tilde V = \tilde{\vec a}_1 \cdot [\tilde{\vec a}_2 \times \tilde{\vec
    a}_3] = \sqrt 2 \, \tilde a^3 \,.
\]
Therefore the density here is
\[
    \tilde \rho
    = \frac{2}{\sqrt 2 \, \tilde a^3}
    = \frac{\sqrt 2}{\tilde a^3} \,.
\]

The problem asks for constant density. The new lattice spacing $\tilde a$ is
desired. So we have
\[
    \rho = \tilde \rho
    \iff
    \frac{2}{a^3} = \frac{\sqrt 2}{\tilde a^3}
    \iff
    \frac{a^3}{2} = \frac{\tilde a^3}{\sqrt 2}
    \iff
    \tilde a
    =
    \sqrt[6] 2 \, a
    \,.
\]

After the transition, the new lattice spacing is a factor $\sqrt[6] 2$ larger
than before.

\section{Reciprocal lattice and Brillouin zone of the kagomé lattice}
\label{homework:2}

\subsection{Reciprocal lattice}

First we need to know the Bravais lattice. The two basis vectors are
\[
    \vec a_1 =
    a
    \begin{pmatrix}
        2 \\ 0
    \end{pmatrix}
    \eqnsep
    \vec a_2 =
    a
    \begin{pmatrix}
        2 \cos(\piup/3) \\
        2 \sin(\piup/3)
    \end{pmatrix}
    =
    a
    \begin{pmatrix}
        1 \\ \sqrt{3}
    \end{pmatrix} \,.
\]
Both vectors have a length of $2a$ which seems consistent.

The reciprocal basis can be found in a convenient way when those vectors are
generalized to $\R^3$ and a third basis vector $a\ev_3$ is added. Then we have
\[
    \vec b_i = 2\piup \frac{\epsilon_{ijk} \vec a_j \times \vec a_k}{V} \,.
\]
The volume~$V$ is
\[
    V = \vec a_1 \cdot [\vec a_2 \times a\ev_3]
    = 2 \sqrt 3 \, a^3 \,.
\]
Now we can compute the reciprocal lattice vectors. They are
\[
    \vec b_1 = \frac{2 \piup}{2 \sqrt 3 \, a^3}
    \begin{pmatrix}
        a^2 \sqrt 3 \\ 0 \\ 0
    \end{pmatrix}
    =
    \frac{\piup}{a}
    \begin{pmatrix}
        1 \\ - 1 / \sqrt 3 \\ 0
    \end{pmatrix}
    \eqnsep
    \vec b_2 = \frac{2 \piup}{2 \sqrt 3 \, a^3}
    \begin{pmatrix}
        0 \\ 2a^2 \\ 0
    \end{pmatrix}
    =
    \frac{\piup}{a}
    \begin{pmatrix}
        0 \\ 2 / \sqrt 3 \\ 0
    \end{pmatrix} \,.
\]

\subsection{First Brillouin zone}

The first Brillouin zone is spanned by those reciprocal vectors. The zone is
centered symmetrically.

\begin{figure}[htbp]
    \begin{subfigure}[b]{0.5\linewidth}
        \centering
        \includegraphics{reciprocal}
        \caption{%
            Reciprocal lattice vectors $\vec b_i$
        }
        \label{fig:}
    \end{subfigure}
    \begin{subfigure}[b]{0.5\linewidth}
        \centering
        \includegraphics{brillouin}
        \caption{%
            First Brillouin zone with marked corners
        }
        \label{fig:}
    \end{subfigure}
    \caption{%
        Construction of the first Brillouin zone
    }
    \label{fig:}
\end{figure}

\subsection{Possible wave-vectors}

\subsection{16-site lattice}


\end{document}

% vim: spell spelllang=en tw=79
